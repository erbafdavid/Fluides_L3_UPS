% !TEX root = TD_fluides_part2_2018.tex


\setcounter{section}{10}


\section{Ecoulements compressibles}


%--------------------------------------------------------------------------------------------------
\subsection{Limite de l'hypoth\`ese incompressible}
%--------------------------------------------------------------------------------------------------

Un tube de Pitot double est plac\'e sur l'axe d'une canalisation de 10 cm
de diam\`etre contenant de l'air en \'ecoulement.
Le manom\`etre diff\'erentiel en U du tube de Pitot est rempli de mercure
de masse volumique $\rho_{Hg}$ = 13 600 kg.m$^{-3}$.
\begin{enumerate}
\item
Quel est le d\'ebit d'air obtenu dans l'hypoth\`ese d'un \'ecoulement
incompressible sachant que la d\'enivellation $\Delta h$ du mercure est
de 10 mm ?
\item
En consid\'erant que les effets de compressibilit\'e peuvent \^etre
n\'eglig\'ees pour des fluctuations de masse volumique de l'air $\rho$
inf\'erieures \`a 1\%,
jusqu'\`a quel nombre de Mach l'\'ecoulement peut \^etre consid\'er\'e
comme incompressible~?
Quelle est alors l'erreur commise sur l'\'evaluation de la vitesse ?
\end{enumerate}

%--------------------------------------------------------------------------------------------------
\subsection{Vol subsonique}
%--------------------------------------------------------------------------------------------------

Un avion vole \`a un nombre de Mach de 0.95 \`a une altitude o\`u la pression
atmosph\'erique est \'egale \`a 0.223 bar et o\`u la masse volumique de l'air
est $\rho=0.349$ kg.m$^{-3}$.
On suppose que l'air se comporte comme un fluide parfait et que les filets
fluides sont thermiquement isol\'es.
\begin{enumerate}
\item
Calculer la vitesse de l'avion en km/h.
\item
Calculer la pression et la temp\'erature au point d'arr\^et sur le bord
d'attaque de l'aile.
\end{enumerate}
 
%--------------------------------------------------------------------------------------------------
\subsection{Ecoulement isentropique}
%--------------------------------------------------------------------------------------------------

Dans un turbo-r\'eacteur, les gaz entrent dans la tuy\`ere \`a la vitesse
de 275 m.s$^{-1}$ et \`a une temp\'erature de 741 $^o$C.
Leur vitesse de sortie est de 564 m.s$^{-1}$.
En supposant le fluide parfait et l'\'ecoulement isentropique, calculer
\begin{enumerate}
\item
la temp\'erature de sortie des gaz,
\item
le nombre de Mach dans les conditions d'\'ejection des gaz.
\end{enumerate}
On assimilera les gaz \`a de l'air pur ($C_p = 1000$ J.kg$^{-1}$.K$^{-1}$).

%--------------------------------------------------------------------------------------------------
\subsection{R\'egimes isentropiques dans une tuyère}
%--------------------------------------------------------------------------------------------------

On consid\`ere dans ce probl\`eme l'\'ecoulement isentropique
d'un gaz parfait dans une tuy\`ere
intercal\'ee entre un r\'eservoir \`a air comprim\'e et l'atmosph\`ere
\`a pression $P_a = 1$ atm.
On assimilera le gaz \`a de l'air pur ($\gamma = 1.405$, 
$r=287$ J/kg/K).
La section de sortie a un diam\`etre $D_s = 5$ cm.

\begin{enumerate}
\setcounter{enumi}{0}
\item[]
\textbf{R\'egime 1 :}
\item[]
Dans ce r\'egime, la vitesse du gaz au niveau du col de la tuy\`ere
est $u_c = 250$ m/s.
La temp\'erature dans le r\'eservoir est $T_i = 300$ K.
\item
Calculer la temp\'erature au col $T_c$
puis le nombre de Mach au col $M_c$.
\item
En d\'eduire la nature (subsonique, supersonique ou sonique)
de l'\'ecoulement dans le convergent, au col et dans le divergent.
\item
\label{it:T_ext}
Par continuit\'e, la temp\'erature en sortie de la tuy\`ere est \'egale
\`a la temp\'erature ext\'erieure ambiante $T_s = T_a = 296$ K.
Calculer le nombre de Mach $M_s$ dans la section de sortie.
\item
\label{it:P_ext}
De la m\^eme fa\c{c}on, la pression en sortie de tuy\`ere $P_s$ correspond
\`a la pression atmosph\'erique $P_a$.
En d\'eduire la pression g\'en\'eratrice $P_i$ dans le r\'eservoir
puis la masse volumique g\'en\'eratrice $\rho_i$.
\item
Calculer la vitesse en sortie $u_s$ et la masse volumique en sortie $\rho_s$
puis en d\'eduire le d\'ebit massique $\dot{m}$ de la tuy\`ere.
\item[]
\textbf{R\'egime 2 :}
\item[]
Dans cette partie, la pression $P_i$ dans le r\'eservoir est augment\'ee
jusqu'\`a ce que le nombre de Mach en sortie atteigne $M_s = 1.5$.
Pour ce r\'egime, on observe ni onde de choc ni onde de d\'etente~:
la tuy\`ere est dite \textit{adapt\'ee}.
\item
D\'eterminer la nature (subsonique, supersonique ou sonique)
de l'\'ecoulement dans le convergent, au col et dans le divergent.
\item
Les conditions thermodynamiques en sortie de tuy\`ere sont les m\^emes
que dans les questions \ref{it:T_ext} et \ref{it:P_ext}.
En d\'eduire la pression g\'en\'eratrice $P_i$ et la temp\'erature
g\'en\'eratrice $T_i$ r\'egnant dans le r\'eservoir \`a air comprim\'e.
\item
	Calculer la temp\'erature au col $T_c$.
\item
	Calculer le diam\`etre de la section au col $D_c$.
\item
	Calculer le d\'ebit massique $\dot{m}$ pour ce r\'egime de fonctionnement de la tuy\`ere.
\end{enumerate}

%==================================================================================================
%\section{Ondes de choc}
%==================================================================================================
