% !TEX root = TD_fluides_part2.tex

\setcounter{section}{6}

%\section{Ecoulements Inertiels II : Ecoulements potentiels}


\section{Ecoulements Inertiels I }


\setcounter{subsection}{-1}



\subsection{Estimation de la traînée d'un cylindre (Exam mai 2017)}


On étudie l'écoulement stationnaire autour d'un cylindre fixe de rayon $a$ et de longueur infinie dans la direction $z$ (écoulement bidimensionnel dans le plan $(x,y)$ ). 
Le fluide est considéré comme un liquide incompressible de masse volumique $\rho$ constante et de viscosité $\nu$. On néglige l'effet de la pesanteur.
On cherche a estimer la force 
${\bf F} = F_x {\bf e_x} + F_y {\bf e_y} $ exercée par le fluide sur le cylindre a partir de bilans intégraux.

%Loin du cylindre, l'écoulement est supposé uniforme, de direction $\myvec{e}_x$ et de valeur %$U_0$, et la pression est également uniforme et vaut $P_0$.


\begin{figure}[htb]
  \begin{center}
      \includegraphics[width=.9\linewidth]{Cylindre_Bilan.png}
      \end{center}
      \vspace{-1cm}
\end{figure}

%La modélisation potentielle des deux parties précédentes ne permet pas de prédire la force de traînée, qui est due aux frottements visqueux à la surface du cylindre. Dans cette partie
%on cherche à estimer cette force de traînée par un bilan intégral dans le cas non tournant ($\Omega = 0$),  a partir de la loi de vitesse observée dans le sillage du cylindre à une distance $L$ de celui-ci.

%On cherche à estimer la force de traînée par un bilan intégral,  
%a partir de la loi de vitesse observée dans le sillage du cylindre à une distance $L$ de celui-ci.
 On considère pour cela un volume de contrôle correspondant à un domaine parallélépipédique 
 de largeur $2b$, longueur $2L$, et profondeur $2H$ (dans la direction z).  Les champ de vitesse ${\bf u} = u {\bf e_x} + v {\bf e_y} $  et de pression sur les frontières de ce volume de contrôle sont indiqués sur la figure. %Les lois $v(x,\pm b)$ et $u(L,y)$ sont 
 La troisième composante de vitesse $w$ est nulle sur toutes les frontières du domaine.
 On suppose que les effets visqueux sont négligeables loin du cylindre (y compris sur les frontières du volume de contrôle).

\begin{enumerate}

\item Ecrire un bilan de masse sur le volume de contrôle, et en déduire une relation intégrale reliant $U_0$, $u(L,y)$, $v(x,-b)$ et $v(x,b)$.
(ici et dans la question suivante on ne cherchera pas à calculer les intégrales apparaissant dans l'expression).


\item Ecrire un bilan de quantité de mouvement (théorème d'Euler) projeté dans la direction $x$, et en déduire une expression de la traînée (composante $F_x$ de la force exercée) sur le cylindre en fonction du champ de vitesse sur les frontières du domaine.


%\item Ecrire également un bilan de masse, et en déduire une relation intégrale reliant $u(L,y)$, 
%$v(x,-b)$ et $v(x,b)$.

\item En combinant les équations obtenues aux deux questions précédentes, montrez que la force de traînée peut être estimée par l'équation suivante :
$
F_x = 2 H \int_{-b}^{b} u(L,y) \left( U_0 - u(L,y) \right) dy 
$


\item
Application. On suppose qu'en sortie le profil de vitesse peut être approximé par la loi suivante: 
$$
u(L,y) = \left\{ \begin{array}{ll} U_0 \quad & |y|> a \\ 0.7 U_0 \quad & |y|< a \end{array}
\right.
$$

En déduire le coefficient de traînée $C_x = F_x / (\rho U_0^2 a H)$.


\item A partir d'un bilan de quantité de mouvement projeté dans la direction $y$, donnez une expression de la portance (composante $F_y$ de la force exercée) en fonction de la vitesse sur les frontières du domaine. A quelle condition cette portance est-elle nulle ? interprétez physiquement.


\end{enumerate}



\subsection{Ski nautique} 

$$
\includegraphics[width= .95 \linewidth]{../FIGURES/SkiNautique2.pdf}
$$
\vspace{-5cm}

Un skieur nautique se déplace à une vitesse $V_0$ (en module) sur une étendue d'eau 
(fluide de propriétés $\rho = 1000 kg/m^3$ et $\nu = 1\cdot 10^{-6} m^2/s$ supposées uniformes) dans l'environnement terrestre au niveau de la mer ($P_{atm} = 10^5 Pa $; $g =10m^2/s$) .

 Il utilise un monoski, supposé rectangulaire, de longueur $L= 1.5m$  et largeur $\ell = 20cm$.

On cherche à estimer la force $\vec F$ exercée sur le ski.

Dans les applications numériques on prendra une valeur de la vitesse $V_0 = 25km/h$.

%\begin{figure}

%\end{figure}


\subsection*{A. Analyse du problème}

\begin{enumerate}

\item A quel régime d'écoulement a-t-on affaire dans ce problème ?
(justifiez en terme de nombres adimensionnels et/ou d'hypothèses physiques).

\rep{Avec les données de l'énoncé le nombre de Reynolds vaut $Re = V_0 L/\nu \approx 10^7$,
donc on est dans un régime inertiel. L'écoulement est de plus incompressible (ou plus exactement isovolume) car l'eau est considéré comme un fluide incompressible (hypothèse justifiée dans la grande majorité des applications ).}


\item On étudie le problème dans le référentiel du skieur, ce qui a l'avantage de pouvoir supposer que l'écoulement d'eau est {\em stationnaire}.
On observe que sous le ski une partie de l'écoulement d'eau est réorienté sous la forme d'un jet d'eau vers l'avant, qui déferle ensuite (voir figure).

Justifiez l'existence d'un point d'arrêt $S$ situé sous le ski. Tracez la forme de la ligne de courant passant par  $S$, ainsi que quelques lignes de courant voisines de celles-ci.

\rep{Une partie de l'eau est réorientée vers l'avant et le reste s'écoule vers l'arrière, il y a donc une ligne de courant qui délimite ces deux régions et celle-ci s'arrête sur la surface inférieure du ski.}

\item On note $[z]$ l'{\em ordre de grandeur des variations d'altitude} dans l'écoulement, et on définit le nombre de Froude par $Fr = \frac{V_0^2}{g [z]} $ Vérifiez que ce nombre est sans dimensions. Quelle est son interprétation physique ?

\rep{ $[Fr] =  \frac{ (m/s)^2}{m/s^2 \cdot m} = 1$. L'interprétation de ce nombre est le rapport entre une énergie cinétique massique et une (variation d')énergie potentielle massique de gravité.
} 

\item En comparant l'ordre de grandeur des différents termes dans l'équation de Bernoulli, montrez que si $Fr \gg 1$ celle-ci se simplifie et conduit à $p+\rho |\vec{u}|^2/2 = cte$. 

\rep{ $[ \rho g z] \approx \rho g [z]$ et $[\rho |\vec{u}|^2/2] \approx \rho V_0^2$, le rapport de ces deux termes et le nombre de Froude.  Si $Fr \gg 1$ le premier terme est négligeable.
}

\item Justifiez qu'il est raisonnable de supposer $[z] \approx \ell$. En tirer les conclusions. 

\rep{L'écoulement est en réalité tridimensionnel avec une largeur caractéristique $\ell$ dans la direction transverse. Au cours de leur mouvement, l'amplitude du déplacement latéral des particules d'eau (d'ordre de grandeur $\ell$) est comparable à celle de leur déplacement vertical 
(d'ordre de grandeur $[z]$).

Avec ces hypothèses le nombre de Froude est $Fr = 24$.

Les variations de pression de nature hydrostatique sont dont négligeables devant celles de nature hydrodynamique.

}


%\item On introduit un point $S'$ situé sur la ligne de courant passant par $S$ et loin en amont du skieur. Donnez la pression $P_S'$ en ce point en fonction de son altitude $y_S'$ comptée a partir du niveau de la mer (attention $y_S'$ est négatif).

%\rep{Dans le référentiel terrestre, la région loin en amont du skieur est au repos, dont un peut appliquer les lois de l'hydrostatique : $P_{S'} = P_{atm} - \rho g z_{S'}$.
%Autre justification : dans le référentiel du skieur, la région située autour du point $S'$ est en écoulement uniforme (donc irrotationnel) à la vitesse $V_0$, donc on peut utiliser Bernoulli sous sa deuxième forme.}

%\item 
%Donnez la valeur de la pression en $S$ ( notée $P_S$) en fonction de $P_{atm}$, $\rho$ et $V_0$, $g$  et des altitudes $z_S$  et $z_S'$ des points $S$ et $S'$

%\rep{On est dans les conditions d'application du thèorème de Bernoulli sur la ligne de courant $SS'$, on en déduit donc $P_{S'} + \rho g z_S' + \rho V_0^2/2 = P_S + \rho g z_S$.
%Donc finalement :
%$$P_s = P_{atm} + \rho V_0^2/2 - \rho g z_S$$}




%\item En supposant $z_S$ et $z_S'$ sont tous deux de l'ordre de grandeur de $\ell$, montrez qu'on peut en fait négliger la gravité dans ce problème. Donnez une nouvelle approximation de  $P_S$ ne dépendant que de  $P_{atm}$, $\rho$ et $V_0$.

%\rep{Si $z_s\approx \ell $, alors $\rho g z_S \approx 2000 U.S.I$ , alors que $\rho V_0^2/2 \approx 24 000 U.S.I$, ce terme est donc négligeable. Donc 
%$$
%P_s = P_{atm} + \rho V_0^2/2
%$$
%}

%\item En supposant que l'ordre de grandeur de la pression sur la surface inférieure du ski est donné par $P \approx P_S$, donnez une estimation de l'ordre de grandeur de la force exercée (par l'eau et par l'air) sur le ski $| \vec{F}|$.

%\item En supposant une vitesse d'ordre $V_0 \approx 25 km/h$, l'estimation précédente est-elle suffisante pour maintenir un skieur hors de l'eau ? Cette estimation vous semble-t-elle réaliste ?


\subsubsection*{B. Estimation de la force par un bilan de quantité de mouvement}

On souhaite maintenant calculer la force à l'aide d'un bilan de quantité de mouvement. 
On note $\alpha$ l'angle d'incidence du ski et on introduit deux repères ($\vec{e}_x, \vec{e}_y$) et 
($\vec{e}_X,\vec{e}_Y$) comme définis sur la figure.



On considère un volume de contrôle $\Omega$ comme représenté sur la figure, délimité par un contour $\partial \Omega = (ABCDEFA)$ comme représenté sur la figure, d'épaisseur $\ell$ (dans la direction perpendiculaire $z$). 
\begin{itemize}
\item  les portions (BC) et (DE) sont des {\em lignes de courant}, 
\item la portion (FA) est la surface du ski sur laquelle on souhaite calculer la force exercée.
On suppose ici que la longueur $(FA)$ vaut $L$.
\item la portion (AB) est une "section d'entrée" de dimension $S_0 = \ell h_0$ à travers laquelle la vitesse est supposée uniforme et
de valeur $\vec{V}_0 = V_0 \vec{e}_x$.
\item la portion (CD) est la section $S_1 = \ell h_1$ à travers laquelle s'effectue le "jet déferlant". On suppose également la vitesse uniforme dans cette section est donnée par $\vec{V}_1 = - V_1 \vec{e}_X$.
\item la portion (EF) est la section "de sortie" $S_2 = \ell h_2$ de l'écoulement. Dans cette section on suppose la vitesse uniforme et donnée par $\vec{V}_2 = + V_2 \vec{e}_X$.
\end{itemize}

\item Justifiez que la pression sur les portions (AB), (BC), (CD), (DE), (EF) il est justifié de considérer que la pression est uniforme et égale à la pression atmosphérique $P_{atm}$ 
(on pourra admettre ce résultat en ce qui concerne la portion (DE) ).

\rep{
\begin{itemize}
\item  La portions (BC) est une {\em surface libre} en contact avec l'air à la pression atmosphérique,
\item A travers la portion (CD), la pression augmente légèrement en fonction de la profondeur :
$p_D = p_C + \rho g (z_b-z_c)$. Cependant l'incrément de pression hydrostatique est négligeable.
\item Le point $F$ est sur une surface libre et donc à pression atmosphérique l'incrément de pression entre $F$ et $E$ est de nature hydrostatique vaut $\rho g (z_E-z_F)$ et est donc négligeable.
\item Raisonnement a peu près identique pour la section (AB).
\item Pour la portion (DE) il semble raisonnable de supposer que la norme de la vitesse reste a peu près constante , c.a.d  $|{\vec u} \approx V_0$, bien que la direction de la vitesse change.
Sous cette hypothèse, Bernoulli nous indique que la pression est uniforme.
\end{itemize}
}

%\rep{Sous la forme générale Bernoulli s'écrit $p+  \rho g z + \rho V^2/2 = cte$. Cependant les variations d'altitude sont toutes d'ordre $\ell$ et on a vu que le terme  $\rho g z $ est alors négligeable devant le terme $\rho V^2/2$.}





\item Montrez que $V_1=V_0$ et $V_2 = V_0$. 

\rep{On peut appliquer Bernoulli sur la ligne de courant CB. $p_B+  \rho V_0^2/2 = p_C+ \rho V_1^2/2$. $P_C = P_B = P_{atm}$, donc $V_1=V_0$. De même en raisonnant sur la ligne de courant  (DE). 
}

\item A l'aide d'un bilan de masse sur le volume de contrôle, donnez une relation entre 
$h_0, h_1$ et $h_2$.

\rep{Le bilan de masse s'écrit : 
$- \int_{\partial \omega} \rho \vec{u} \cdot \vec{n} dS = 0$ où $\vec{n}$ est la normale sortante.

Sur AB : $\vec{n} = -\vec{e}_x$ et $\vec{u} = V_0 \vec{e}_x$.
Sur CD : $\vec{n} = -\vec{e}_X$ et $\vec{u} = -V_0 \vec{e}_x$.
Sur EF : $\vec{n} = \vec{e}_X$ et $\vec{u} = V_0 \vec{e}_X$.

Sur les autres portions le débit est nul. Seules les intégrales sur les portions AB, CD et EF du contour sont pertinentes, et on arrive à :
$$ - \rho (h_1 \ell) V_1 + \rho (h_0  \ell) V_0 -\rho (h_2  \ell) V_2 = 0$$
Compte tenu du fait que $V_1 = V_2 = V_0$ et en simplifiant par $\rho \ell$ on arrive à :
$$
h_0 = h_1+h_2
$$
}



\item A l'aide du bilan de quantité de mouvement sur le volume de contrôle (théorème d'Euler), donnez une expression de la force $\vec{F}$ sous forme d'une intégrale sur le contour (ouvert) $(ABCDEF)$.

\rep{
Le théorème d'Euler s'écrit :
$$
\vec{F}_{eau->(FA)} = - \int_{(ABCDEF)} \left( p \vec{n} + \rho \vec{u} ( \vec{u} \cdot \vec{n} ) \right) dS
$$
}




\item En projetant cette expression sous la forme 
$\vec{F}  = F_X \vec{e_X} + F_Y \vec{e_Y} $, et en simplifiant les intégrales compte tenu des hypothèses formulées, montrez que :
$$
F_X = \rho \ell V_0 ( h_0 \cos \alpha + h_1 - h_2 )
$$
et donnez une expression similaire pour la composante $F_Y$ .

\rep{En développant l'intégrale précédente :
$$ 
\vec{F} = - \int_{(AB)} \rho \vec{u} ( \vec{u} \cdot \vec{n} ) dS - \int_{(CD)} \rho V_0^2 \vec{e_x} dS  
- \int_{(CD)} \rho V_0^2 \vec{e_x} dS
- \int_{(ABCDEF)} \left( p_{atm} \vec{n}  \right) dS
$$
$$
=
\rho \ell h_0 V_0^2  \vec{e_x} - \rho \ell h_1 V_1^2  \vec{e_X} +  \rho \ell h_1 V_1^2  \vec{e_X} 
+ L \ell p_{atm} \vec{e}_Y   
$$


où le terme de pression a été simplifié en tenant compte de l'identité $\int_{\partial \Omega}  p_{atm} \vec{n}  dS = \vec{0}$, et donc 
$
- \int_{(ABCDEF)} \left( p_{atm} \vec{n}  \right) dS = + \int_{(FA)} \left( p_{atm} \vec{n}  \right) dS
$ 
avec $\vec{n} = \vec{e}_Y$.

En projetant dans les axes :
$F_X = \rho \ell V_0^2 (h_0 \cos \alpha + h_1 - h_2)$

$F_Y = \rho \ell V_0^2 h_0 \sin \alpha  + L \ell p_{atm}$

}


\item Justifiez pourquoi la composante $F_X$ est nulle. En déduire que $h_1 = h_0 (1- \cos \alpha)/2$ et donnez une expression similaire pour $h_2$.

\rep{
Sous les hypothèses du régime inertiel, la force exercée sur la surface du ski est due uniquement à la pression (contrainte visqueuse négligée) et s'exerce donc uniquement dans la direction normale $\vec{e}_Y$.

En combinant les équations $F_X = 0$ et $h_0 = h_1+h_2$ on arrive au résultat demandé.
 
}

\item Montrez que la composante selon $Y$ des forces exercées sur le ski par les fluides (eau+air) est :
$$
F_{Y,(eau+air)} = \rho \ell h_0 V_0 \sin \alpha  
$$

\rep{
La composante de la force exercée par l'eau a été calculée ci-dessus. La force exercée par l'air vaut $\vec{F}_{(air)} = - L \ell p_{atm} \vec{e}_Y$. En ajoutant les deux on trouve le résultat demandé.
}

\item 
On écrit maintenant la force sous la forme $\vec{F}  = F_y \vec{e_y} + F_y \vec{e_y} $. Exprimez la traînée $F_x$ et la portance $F_y$ en fonction des paramètres du problème.
\rep{
$F_x = \rho \ell h_0 V_0 \sin^2 \alpha ; \quad 
F_y = \rho \ell h_0 V_0 \sin \alpha \cos \alpha =  \rho \ell h_0 V_0 / 2 \sin 2 \alpha$
}

\item Applications numériques : En supposant que $h_0 \approx \ell$  et $V_0 = 25 km/h$, donnez l'angle $\alpha$ générant une portance équilibrant le poids d'un skieur d'une masse de $80kg$.
Que vaut alors la traînée $F_x$ .


\end{enumerate}


%\subsection{Vidange d'un réservoir}

%{\em Exercice préparatoire ; énoncé et correction détaillée sur moodle}
\comment{
%--------------------------------------------------------------------------------------------------
\subsection{Force exerc\'ee par un jet sur une plaque}
%--------------------------------------------------------------------------------------------------

\begin{figure}[hbtp]
  \begin{center}
    \setlength{\unitlength}{1mm}
    \begin{picture}(80, 60)(0, 0)
      \put(10, 0){\includegraphics[width=6cm]{jet_impact.pdf}}
      \put(20, 40){$P_a$}
      \put(60, 40){$y$}
      \put(65, 20){$x$}
      \put(72, 30){$X$}
      \put(20, 40){$P_a$}
      \put(0, 27){$\vec{V}, \, q_m$}
      \put(50, 60){$\vec{V}', \, q_m'$}
      \put(10, 0){$\vec{V}'', \, q_m''$}
    \end{picture}
  \end{center}
  \mycaption{Jet plan impactant sur une plaque inclin\'ee.}
  \label{fig:jet_impact}
\end{figure}

Un jet d'eau bidimensionnel s'\'ecoulant \`a vitesse $\vec{V}$ (d\'ebit $q_m$) impacte 
une plaque plane inclin\'ee d'un angle $\alpha$ par rapport \`a
l'horizontale $X$ (fig.~\ref{fig:jet_impact}). 
On veut d\'eterminer la force exerc\'ee par le jet sur la plaque.
On note $q_m'$ et $q_m''$ les d\'ebits correspondant aux vitesses
$\vec{V}\,'$ et $\vec{V}\,''$ de l'\'ecoulement de part et d'autre
de l'impact.
On supposera les effets visqueux n\'egligeables, ainsi que l'effet de la pesanteur.
\begin{enumerate}
\item 
  Montrer que $V = V' = V''$.
\item 
  Exprimer la conservation du d\'ebit massique.
\item 
  D\'eterminer la force $\vec{F}$ exerc\'ee par les fluides sur la plaque.
  On exprimera les composantes $F_x$, $F_y$ et $F_X$.
\item 
  Calculer $q_m'$ et $q_m''$ en fonction de $q_m$ et $\alpha$.
\item 
  A quelles conditions les effets visqueux sont-ils effectivement n\'egligeables ?
\end{enumerate}
}



%--------------------------------------------------------------------------------------------------
\subsection{Ventilation d'un tunnel (d'apr\`es examen 2008)}
%--------------------------------------------------------------------------------------------------

On consid\`ere un tunnel horizontal de section $S$ uniforme, ventil\'e par un soufflage d'air 
frais comme indiqu\'e sur la figure~\ref{fig:tunnel}. 
L'air souffl\'e \`a la vitesse $U_s$ dans un conduit de section $s$ se m\'elange \`a l'air 
du tunnel entre les sections 1 et 2. 
Dans la partie amont du tunnel (partie gauche), le soufflage peut induire un \'ecoulement 
vers la droite comme sur la figure, ou vers la gauche. On veut d\'eterminer les conditions 
r\'ealisant l'un ou l'autre \'ecoulement. 
Dans tout le probl\`eme, on n\'egligera les effets visqueux, et on consid\`erera l'\'ecoulement
isovolume, stationnaire, et uniforme dans toute section du tunnel, sauf entre les sections 1 et 2.

\begin{figure}[hbt]
  \begin{center}
    \setlength{\unitlength}{1mm}
    \begin{picture}(150, 40)(0, 5)
      \put(0, 0){\includegraphics[width=15cm]{tunnel_ventile.png}}
      \put(0, 37){$p_a$} %, u = 0
      \put(145, 20){$p_a$}
      \put(65, 23){$s$}
      \put(50, 8){$S$}
      \put(78, 26){$\alpha$}
      \put(85, 12){$U_s$}
      \put(62, 11){$U_1$}
      \put(99, 11){$U_2$}
    \end{picture}
  \end{center}
  \mycaption{Sch\'ema du tunnel ventil\'e.}
  \label{fig:tunnel}
\end{figure}

\begin{enumerate}

\item 
  On consid\`ere que l'\'ecoulement correspond \`a la figure : 
  au voisinage de l'entr\'ee du tunnel, l'air atmosph\'erique est acc\'el\'er\'e progressivement, 
  alors qu'il sort du tunnel sous la forme d'un jet.
  \begin{enumerate}
  \item 
    Ecrire la relation de Bernoulli pour une ligne de courant entre l'ext\'erieur du tunnel 
    (infini amont) et la section 1.
  \item 
    Quelle est la pression de l'air \`a la sortie du tunnel~? 
    En d\'eduire, en utilisant la relation de Bernoulli, la pression dans la section 2.
  \item 
    Appliquer le bilan int\'egral de quantit\'e de mouvement pour 
    le volume de contr\^ole limit\'e par les parois du tunnel, les sections 1 et 2 et 
    la section de soufflage, d'aire $s/\sin \alpha$ o\`u $\alpha$ est l'angle du conduit 
    de soufflage avec l'axe du tunnel. 
    On consid\`erera l'angle $\alpha$ petit, soit $\cos \alpha \approx 1$.
  \item 
    Ecrire la relation liant $U_s$, $U_1$ et $U_2$, issue de la conservation de la masse.
  \item 
    D\'eduire des \'equations \'etablies ci-dessus la relation quadratique donnant 
    la vitesse normalis\'ee $u = U_1/U_s$ en fonction du rapport des sections $r = s/S$. 
  \item 
    Montrer qu'il n'existe de solution correspondant \`a la figure que si le rapport des 
    sections $r$ est inf\'erieur \`a une valeur critique que l'on pr\'ecisera.
  \item 
    Tracer l'\'evolution de $u$ en fonction de $r$, en pr\'ecisant les points remarquables 
    de la courbe. On donnera en particulier la valeur de $r$ qui maximise l'entra\^{\i}nement 
    d'air, et la valeur de ce maximum de $u$.
  \end{enumerate}
  
\item 
  Lorsque le rapport des sections exc\`ede la valeur critique d\'etermin\'ee ci-dessus, 
  l'\'ecoulement dans la partie amont du tunnel s'inverse. 
  Aux deux extr\'emit\'es du tunnel, l'air sort sous la forme d'un jet.
  \begin{enumerate}
  \item 
    En proc\'edant de fa\c{c}on similaire \`a la question pr\'ec\'edente, 
    d\'eterminer la vitesse normalis\'ee $u$ en fonction du rapport des sections $r$.
  \item 
    V\'erifier que la vitesse $u$, en tant que fonction de $r$, ne subit pas de discontinuit\'e 
    entre les deux r\'egimes d'\'ecoulement.
  \item 
    Pour un tunnel de 6 m\`etres de diam\`etre et des vitesses d'\'ecoulement de l'ordre 
    du m\`etre par seconde, le fait de n\'egliger les effets visqueux vous para\^{\i}t-il 
    justifi\'e ?
  \end{enumerate}
  
\end{enumerate}






\comment{
%--------------------------------------------------------------------------------------------------
\subsection{Attirer en soufflant \exonormal}
%--------------------------------------------------------------------------------------------------

On consid\`ere deux disques parall\`eles, de diam\`etre $D$, s\'epar\'es d'une distance $d \ll D$ (fig.~\ref{fig:interdisques}). De l'air, amen\'e par un tube normal aux disques avec une vitesse $V_c$, p\'en\`etre au centre du disque sup\'erieur par une ouverture de diam\`etre $d \ll D$. Au-del\`a de la section d'entr\'ee, l'air s'\'ecoule radialement  pour d\'eboucher dans l'atmosph\`ere. On n\'eglige les effets visqueux. 

\begin{figure}[htb]
\begin{center}
\includegraphics[width=8cm]{interdisques}
\end{center}
\mycaption{Ecoulement inter disques.}
\label{fig:interdisques}
\end{figure}

On notera $r$ la distance d'un point \`a l'axe $z$ des deux disques. Dans tout le probl\`eme, on admet que le module de la vitesse entre les deux disques ne d\'epend que de $r$ (sauf dans la zone $r<d/2$ o\`u les lignes de courant changent de direction). Dans la section de sortie $r=D/2$, la pression est \'egale \`a la pression atmosph\'erique $P_a$.

\begin{enumerate}
\item Calculer la vitesse $V(r)$ entre les deux disques pour $d/2<r<D/2$.
Comparer avec $V_c$.
\item Donner la valeur de la pression $P_c$ dans la conduite en fonction
de $V_c$ et $P_a$.
\item D\'eterminer la loi de pression $p(r)$ entre les deux disques pour $d/2<r<D/2$.
Comparer avec $P_a$.
\item Calculer la r\'esultante des forces subies par le disque sup\'erieur de la part
des fluides qui l'entourent pour $d/2<r<D/2$.
Y a-t'il r\'epulsion ou attraction~?
\item On surestime la pression aux points du disque inf\'erieur $r<d/2$ 
en la prenant \'egale \`a la pression en $0$.
Reprendre la question pr\'ec\'edente dans le cadre de cette hypoth\`ese.
\end{enumerate}


%--------------------------------------------------------------------------------------------------
\subsection{Lance d'incendie (d'apr\`es examen 2005) \exonormal}
%--------------------------------------------------------------------------------------------------

La lutte contre les incendies hors zone urbaine et particuli\`erement en
zone dite ``s\`eche'' b\'en\'eficie de l'aide d'appoint d'unit\'es mobiles
transportant un syst\`eme autonome constitu\'e d'un r\'eservoir d'eau,
d'une pompe et d'une lance \`a incendie embarqu\'es
sur un v\'ehicule tout terrain.
On s'int\'eresse dans ce probl\`eme \`a la description de certains aspects
du fonctionnement de ce syst\`eme.

L'eau est mise sous pression par un compresseur qui impose une pression
en entr\'ee de lance de $P_1 = P_0 + \varphi$ o\`u $P_0 = 10^5$ Pa 
d\'esigne la pression atmosph\'erique ambiante et $\varphi > 0$ la surpression,
de l'ordre de quelques bars.
La lance \`a incendie est reli\'ee au compresseur par un tuyau flexible,
et est maintenue par un support orientable.
Le diam\`etre d'entr\'ee de la lance est $D = 5$ cm, correspondant
\`a une section $S_1$.
La lance forme une conduite convergente, de section de sortie $S_0<S_1$. La surface lat\'erale de la lance est not\'ee $\Sigma$.

On rappelle que la masse volumique de l'eau est $\rho = 10^3 \, \mbox{kg/m}^3$
et sa viscosit\'e cin\'ematique $\nu = 10^{-6} \, \mbox{m}^2\mbox{/s}$.

\begin{figure}[htbp]
\begin{center}
\setlength{\unitlength}{1mm}
\begin{picture}(110, 30)
\put(0, 0){\includegraphics[width=10cm]{lance.pdf}}
\put(100, 10){$\vec{e}_x$}
\end{picture}
\end{center}
\mycaption{Mod\`ele de lance \`a incendie.}
\label{fig:lance}
\end{figure}

\begin{enumerate}
\item[]
\item
En supposant que l'ensemble reste globalement 
\`a un m\^eme niveau horizontal pendant le fonctionnement
de la lance, montrer que l'on peut n\'egliger l'influence de la pesanteur
dans cette configuration.
%\item[]
\item[]
Dans un premier temps, la lance d'incendie est ferm\'ee en $S_0$ :
il n'y a donc pas \'ecoulement.
\item
Calculer la r\'esultante $\vec{F}_0$ des forces de pression 
exerc\'ees sur la section
$S_0$ par l'eau sous pression dans la lance et l'air \`a pression
atmosph\'erique, en fonction de $\varphi$ et $S_0$.
\item
Calculer la r\'esultante $\vec{F}_1$ des forces de pression qui s'exercent
cette fois-ci sur la totalit\'e de la paroi de la lance
$S_0 \cup \Sigma$ en fonction de $\varphi$ et $S_1$.
\\
On pourra utiliser sans la red\'emontrer l'\'egalit\'e math\'ematique
suivante :
\[
\int\!\!\!\int_{\partial \Omega} \vec{n} \, \mbox{dS} = \vec{0}
\]
o\`u $\partial \Omega$ d\'esigne une surface \textit{ferm\'ee} de normale
sortante $\vec{n}$.
\item
En d\'eduire la force $\vec{R}$ 
exerc\'ee par l'ensemble \{tuyau et support\} sur la lance.
\item[]
%\item[]
On ouvre maintenant la lance en $S_0$ : un jet d'eau sort \`a vitesse $U_0$
dans l'air \`a pression atmosph\'erique $P_0$.
\item
Calculer la vitesse $U_1$ en entr\'ee de la lance et
la vitesse $U_0$ en sortie, en fonction de $\varphi$, $\rho$ et du rapport
des section $\alpha = S_1/S_0 > 1$ (on admettra que la vitesse est uniforme dans toute section).
\item
Donner alors l'expression du nombre de Reynolds $Re$ en entr\'ee de la lance
d'incendie en fonction en particulier de $\varphi$. \\
Application num\'erique : calculer son ordre de grandeur pour un rapport
de section $\alpha=2$ et un r\'egime de fonctionnement correspondant \`a
$\varphi = 2P_0$.
En d\'eduire la nature laminaire ou turbulente du r\'egime d'\'ecoulement
dans la lance. L'hypoth\`ese de vitesse uniforme faite \`a la quesion pr\'ec\'edente est-elle justifi\'ee ?
\item
Donner l'expression du d\'ebit massique $\dot{m}$ puis du d\'ebit volumique $q$
et calculer leur valeur dans le r\'egime de fonctionnement d\'ecrit dans
la question pr\'ec\'edente.
\item
Donner l'expression de la r\'esultante $\vec{F}$
des forces de pression exerc\'ees par les fluides (eau et air)
sur la paroi $\Sigma$ de la lance, en fonction de $q$, $\rho$, $\alpha$
et $S_1$.
Dans quel sens s'exerce cette force~?
\item
Comparer cette force avec la force $\vec{R}$ exerc\'ee par le tuyau
et le support dans le cas sans \'ecoulement (on pourra pour ce faire
\'ecrire cette derni\`ere force en fonction de $q$, $\rho$, $\alpha$
et $S_1$).
Que pourrait-on en d\'eduire sur le mouvement de la lance au moment
de la mise en \'ecoulement \`a l'ouverture de $S_0$~?
\item
En consid\'erant un bilan int\'egral de quantit\'e de mouvement pour l'ensemble
du syst\`eme embarqu\'e, quelle est la force induite par le jet d'eau sur
le v\'ehicule d'intervention~? 
\end{enumerate}
}




\subsection{Tube de pitot}


\begin{figure}[htb]
  \begin{center}
      \includegraphics[width=.7\linewidth]{Pitot.pdf}
      \end{center}
      \vspace{-1cm}
\end{figure}

{\em Exercice complémentaire ; Correction disponible sur l'espace Moodle du cours de Mécanique des fluides de L2 (TD6, exercice 9)}.

Pour mesurer la vitesse $U_0$ de vehicules se déplaçant à grande vitesse dans l'air (masse volumique de l’air $\rho_a = 1.225kg/m^3$ ),  on utilise généralement des sondes de Pitot. Dans le manomètre en U du tube de Pitot, on utilise un liquide de masse volumique connue, de l'eau par exemple ($\rho_e = 1000kg.m^3$). On prendra $g = 9.81 m/s^2$.

La dénivellation indiquée dans le manomètre en U étant $h =10 cm$ , calculer la vitesse $U_0$.




\comment{

%%%%%%%%%%%%%%%%%%%%%%%%%%%%%%%%%%%%%%%%%%%%%%%%%%%%%%%%%%%%%%%%%%%%%%%%%%%%%%%
\subsection{Siphon}
%%%%%%%%%%%%%%%%%%%%%%%%%%%%%%%%%%%%%%%%%%%%%%%%%%%%%%%%%%%%%%%%%%%%%%%%%%%%%%%

\noindent
\begin{tabular}{lp{1cm}r}
\begin{minipage}{7cm}
Un siphon permet l'\'ecoulement de l'eau d'un r\'eservoir
de grandes dimensions.
Il est constitu\'e d'un tuyau de diam\`etre constant qui s'\'el\`eve
\`a une hauteur $H$ au-dessus du niveau de la surface libre.

Calculer la valeur du d\'ebit maximum que l'on peut obtenir avec ce dispositif
sans qu'il se produise de cavitation (pression quasi nulle en un point
du siphon~: dans ce cas l'eau devient gazeuse et les bulles de gaz ainsi
g\'en\'er\'ees, plus l\'eg\`eres que le liquide, viennent se loger
en haut du siphon, provoquant un risque de d\'esamor\c{c}age).
Quelle est alors l'altitude de la sortie du siphon ?
\end{minipage}
& &
\begin{minipage}{7cm}
\begin{center}
%\input{Figures_FluidesL2/siphon.pstex_t}
\end{center}
\end{minipage}
\end{tabular}


%%%%%%%%%%%%%%%%%%%%%%%%%%%%%%%%%%%%%%%%%%%%%%%%%%%%%%%%%%%%%%%%%%%%%%%%%%%%%%%
\subsection{D\'ebit d'un barrage}
%%%%%%%%%%%%%%%%%%%%%%%%%%%%%%%%%%%%%%%%%%%%%%%%%%%%%%%%%%%%%%%%%%%%%%%%%%%%%%%

Une vanne retient partiellement l'eau d'un barrage de grandes dimensions
dans lequel la hauteur d'eau, fix\'ee, est $H$.
Calculer la hauteur $h$ de sortie correspondant au d\'ebit maximum.

\begin{figure}[hbt]
\begin{center}
%\input{Figures_FluidesL2/barrage.pstex_t} \hskip 2cm \input{../FIGURES/clepsydre.pstex_t}
\end{center}
\caption{(a) Barrage. (b) Clepsydre.} 
\label{fig:baclep}
\end{figure}

%%%%%%%%%%%%%%%%%%%%%%%%%%%%%%%%%%%%%%%%%%%%%%%%%%%%%%%%%%%%%%%%%%%%%%%%%%%%%%%
\subsection{Clepsydre}
%%%%%%%%%%%%%%%%%%%%%%%%%%%%%%%%%%%%%%%%%%%%%%%%%%%%%%%%%%%%%%%%%%%%%%%%%%%%%%%

La clepsydre est une horloge \`a eau connue aussi bien des Egyptiens que des
Am\'erindiens ou des Grecs.
Si le cadran solaire donne l'heure pendant le jour, 
la clepsydre fait la m\^eme chose la nuit, et elle mesure en plus
des dur\'ees plus br\`eves avec une bonne pr\'ecision. 
Un vase perc\'e d'un trou (section $S_0$) laisse couler de l'eau.
Des graduations situ\'ees \`a l'int\'erieur permettent de mesurer des
intervalles de temps.
On souhaite d\'eterminer la forme du vase
telle que la surface libre du liquide, d'aire $S(z)$, s'abaisse 
proportionnellement au temps.
\begin{enumerate}
\item
Calculer la vitesse d'\'ejection de l'eau en sortie ($z=0$).
\item
Donner la loi de variation de l'aire $S(z)$ de la clepsydre.
\item
Dans le cas o\`u le r\'eservoir est de section circulaire,
en d\'eduire la loi de variation de $R(z)$.
\end{enumerate}

%\newpage

%\setlength{\oddsidemargin}{-1cm}
%\setlength{\evensidemargin}{-1cm}


%%%%%%%%%%%%%%%%%%%%%%%%%%%%%%%%%%%%%%%%%%%%%%%%%%%%%%%%%%%%%%%%%%%%%%%%%%%%%%%
\subsection{Ecoulements \`a surface libre}
%%%%%%%%%%%%%%%%%%%%%%%%%%%%%%%%%%%%%%%%%%%%%%%%%%%%%%%%%%%%%%%%%%%%%%%%%%%%%%%

\begin{enumerate}
\item
Une nappe d'eau s'\'ecoule sur un sol qui pr\'esente une d\'enivellation $H>0$
(bosse).
En d\'esignant par $h_1$ et $V_1$ la hauteur d'eau et la vitesse \`a l'infini
amont, montrer que $h_2 > h_1$ correspond \`a $V_1>\sqrt{gh_1}$
(r\'egime torrentiel), et r\'eciproquement
$h_2 < h_1$ si $V_1<\sqrt{gh_1}$ (r\'egime fluvial),
o\`u $h_2$ est la hauteur d'eau \`a l'aplomb de la d\'enivellation.
On notera $V_2$ la vitesse de l'\'ecoulement juste au-dessus de la
d\'enivellation, et on supposera que l'\'ecoulement est uniforme dans chaque
section.
\item
Reprendre l'exercice pr\'ec\'edent et retrouver les deux r\'egimes
d'\'ecoulement en laissant le fond horizontal, mais en imposant une diminution
de la section $e$ de passage par un r\'etr\'ecissement du canal
(\'ecoulement entre les piles d'un pont par exemple).
\end{enumerate}

\begin{figure}[hbt]
\begin{center}
%\input{Figures_FluidesL2/torrent.pstex_t} \hskip 2cm \input{Figures_FluidesL2/pont.pstex_t}
\end{center}
\caption{Ecoulements \`a surface libre.}
\label{fig:esl}
\end{figure}

%%%%%%%%%%%%%%%%%%%%%%%%%%%%%%%%%%%%%%%%%%%%%%%%%%%%%%%%%%%%%%%%%%%

\newpage
\subsection{Le diffuseur de parfum }


%\input{Figures_FluidesL2/parfum.pstex_t}


Le dispositif représenté sur la figure est utilisé pour diffuser du parfum.
Un courrant d'air, de débit volumique $q$ constant, circule dans un conduit
présentant un rétrécissement. Un tuyau, raccordé au récipient contenant le 
parfum, est disposé au col. Si le débit d'air est suffisant, le parfum
est aspiré dans le tuyau, et circule avec un débit volumique $q_l$.
On suppose le débit volumique du parfum faible devant celui de l'air,
de sorte que le débit volumique en sortie est $q+q_l \approx q$.

On note $\rho_a$ et $\rho_l$ les masses volumiques de l'air et du liquide,
$S$ et $S_c$ les sections en entrée et au col, et $\Delta h$ la hauteur entre
la surface du liquide dans le récipient et la sortie du tuyau.

1/ Montrez qu'un débit d'air minimal est nécessaire pour que le parfum
soit diffusé.
Lorsque le débit d'air est inférieur à cette valeur, donnez la hauteur atteinte
dans le tube par le parfum.

2/ Lorsque le débit d'air est supérieur à la valeur critique, calculez
le débit volumique de parfum diffusé.

3/ Représentez la fraction volumique de parfum, $q_l/q$, en fonction de $q$.



Remarque : Un dispositif du même type est utilisé dans certains carburateurs
(mais ca sent moins bon).

}
