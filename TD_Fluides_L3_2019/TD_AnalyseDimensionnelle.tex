% !TEX root = TD_fluides_part1.tex


%==================================================================================================
\section{Analyse dimensionnelle et similitude}
%==================================================================================================

\subsection{Force de résistance à l'avancement d'un bateau de pêche}

%\section{Résistance à l'avancement d'un pétrolier}

On cherche à déterminer la résistance à l'avancement d'un petit bateau de pêche naviguant dans de l'eau de mer à la vitesse nominale $U = 6 m/s$
La longueur du navire est de $L = 10 m$, et sa surface mouillée est de $S = 37,64 m^2$. 

%La masse volumique de l'eau vaut $\rho = 1026 kg/m^3$ pour le navire. La viscosité cinématique de l'eau est $\nu = 1.188 10^{-6} m /s^2$.

\subsubsection{Analyse dimensionnelle}

\begin{enumerate}

\item Listez tous les paramètres physiques pertinents et indépendants ayant une influence sur la force de résistance.

\item En déduire que la résistance à l'avancement peut se mettre sous la forme suivante :

$$ 
R_T = \frac{1}{2} \rho S U^2 \cdot  C_T \left( \frac{U L}{\nu} , \frac{U}{\sqrt{gL}} \right) 
$$  

\item Quels nombres adimensionnels classiques reconnaissez-vous là ? quelle est leur interprétation ?

\item Calculez la valeur de ces deux nombres dans le cas du bateau de pêche en condition réelles.

\rep{ Réponse : $Fr = 0.4$, $Re = 3.54e 7$}  


\subsubsection{Etude de similitude}

On cherche à déterminer la traînée à l'aide d'une expérience en bassin de traction (dans l'eau douce), à l'aide d'une maquette à l'échelle 1/10ème.

\item Est-il possible de faire une expérience en similitude totale (nombres de Froude et de Reynolds identiques dans l'expérience et le navire réel) ? 

\item Quelle doit être la vitesse $U_m$ de la maquette si l'on souhaite respecter la similitude de Reynolds ? Que vaut alors le nombre de Froude de l'expérience ? ce cas vous semble-t-il réaliste ?

\item Quelle doit être la vitesse $U_m$ de la maquette si l'on souhaite respecter la similitude de Froude ? Que vaut alors le nombre de Reynolds $Re_m$ de l'expérience ?

\rep{ $U_m = 1.26 m/s$ ; $Re_m = 1.072e6$. } 

\subsubsection{Similitude partielle sous l'hypothèse de Froude} 

 Lorsque $Re \gg 10^{5}$ et $Fr \le 0.4$, l'expérience montre que l'on peut faire l'hypothèse de Froude, qui consiste à supposer que la  résistance à l'avancement se décompose en deux parties données par les expressions suivantes :

$$
C_T(Re, Fr) = C_{f,0}(Re) + C_{w}(Fr) 
$$ 

où $C_v$ est le coefficient de traînée visqueuse, estimé par la loi empirique de Hugues :
$$
C_v(Re) = \frac{0.074}{\left( \log_{10} Re - 2 \right)^2}
$$

et $C_{w}(Fr)$ est la traînée de vagues, qui peut être mesuré en effectuant une expérience en similitude de Froude.

\item  Pour une vitesse d'avance de la maquette correspondant à celle calculée à la question 7, on mesure sur la maquette une résistance à l'avancement 
$C_{T,m}  = 1.267 N$. 

En déduire la traînée totale exercée sur le navire réel $C_T$, puis la puissance qui doit être fournie par le moteur du bateau.

\rep{Rep : le coefficient de traïnée de la maquette vaut $C_{T,m} = 0.051$. En retranchant la partie visqueuse donnée par la loi de Hugues on déduit $C_w =  0.003$. 
Donc le coefficient de traïnée du navire réel vaut $C_T = 0.003 + C_v(Re) = 0.0043$, et la traïnée totale vaut 
$R_T = 1329 N$.
La puissance correspondante vaut $R_T U = 5173 W= 7.23$ chevaux-vapeur. 
}


\end{enumerate}



\subsection{Force de traînée sur une automobile}

On cherche à estimer la force de traînée $D$ s'exerçant sur une automobile de longueur $L$ et surface frontale $S$ roulant à la vitesse $V$, à l'aide d'une expérience utilisant une maquette à échelle réduite dans un tunnel hydrodynamique.

\begin{enumerate}
\item Listez et discutez les paramètres physiques pertinents et indépendants dans ce problème.  

%{\em Réponse : $g$ n'est pas pertinent (contribue (faiblement) à la force verticale via la poussée d'Archimède mais pas à la force horizontale), $p_{atm}$ non plus (incompressible car faible Mach), 
%$S$ non plus (proportionnel à $L^2$ pour une géométrie de voiture donnée).
 %Donc $ D= { \cal F} ( V, L, \rho, \nu)$.
%}

\item En utilisant les principes de l'analyse dimensionnelle, montrez que la traînée s'exprime simplement sous la forme suivante :

$$ 
D = \rho S V^2 C_x( Re),  \quad \mbox{ avec } Re = \frac{V L}{\nu}
$$

%{\em Réponse : l'analyse dimensionnelle donne $D/(\rho L^2 V^2) = f(V L/\nu)$, par convention on préfère utiliser la surface de référence $S$ (proportionnelle à $L^2$ pour une géométrie donnée).}


\item Calculez la valeur du nombre de Reynolds, pour une automobile de longueur $L = 4m$ (et surface frontale $S = 1.74 m^2$), roulant à la vitesse $V= 90 km/h$. 

%{\em Réponse : $Re = 1.66 \cdot 10^6$.}

\item Dans l'expérience, la maquette est à l'échelle 1/10 (c.a.d. $L_m = L/10$). La vitesse de l'eau dans le tunnel hydrodynamique est notée $V_m$.

Quelle doit être la valeur de $V_m$ pour assurer la similitude de Reynolds ?

%{\em Réponse : $Re = Re_m$ donc $V_m = \nu_m /L_m Re = 16,6 m/s$}

\item 
Le tunnel hydrodynamique étant réglé à la vitesse $V_m$ précédemment déterminée, 
on mesure à l'aide d'une balance de forces une traînée sur la maquette $D_m = 911 N$.
En déduire le $C_x$, puis la traînée $D$ sur la voiture.

%{\em Réponse : $C_x = 0.38$ ; $D = 260 N$.}

\item En déduire la puissance dissipée par les forces aérodynamiques, puis la puissance qui doit être fournie par le moteur, en supposant que le rendement énergétique global de celui-ci est $\eta = 20 \%$.

%{\em Réponse : $DV = 6.5 kW ; \cal{P} = 65 kW = 87 hp$. ($1 hp$ = 1 cheval-vapeur = $746 W$; unité différente du cheval-fiscal apparaissant sur la carte grise !)
%}

\end{enumerate}
 
 
%{\em Remarque : je ne donne pas les valeurs de la masse volumique et viscosité de l'air et de l'eau, c'est intentionnel, normalement ils devraient les connaitre...}



%\subsection{Traînée d'un bateau}


\subsection{Vitesse d'un animal volant}

On cherche à estimer la vitesse $V$ d'un animal volant (insecte, oiseau ou chauve-souris) en fonction de sa masse $M$.


%\subsubsection{Analyse dimensionnelle}


\begin{enumerate}

\item 

Listez les différents paramètres physiques intervenant dans le problème, et discutez leur pertinence. Montrez qu'il est légitime de supposer que la vitesse est donnée par une loi de la forme 

$$
V = {\cal F}(M,L,\rho,g,\nu).
$$

où $\rho$ est la masse volumique de l'air, $g$ l'accélération de la gravité,
$L$ l'envergure de l'oiseau, $M$ sa masse et $\nu$ la viscosité cinématique de l'air.


%\item Commentez l'hypothèse faite ci-dessus ; quels paramètres physiques 
%n'ont pas été pris en compte et comment justifiez-vous leur omission ?

\item Donnez les dimensions physiques de $V$, $\rho$, $\nu$, $g$, $L$ et $M$.

\item En appliquant les principes de l'analyse dimensionnelle, montrez
que la loi de vitesse peut se mettre sous la forme :

$$
V = \sqrt{gL} \,\, {\cal F} \left(\frac{M}{\rho L^3},\frac{\nu}{\sqrt{g L^3}}\right).
$$

 \item Interprétez physiquement la quantité $M/ (\rho L^3)$. Justifiez pourquoi
on peut supposer en première approximation que cette quantité a la même valeur
pour tous les animaux volants.
 
\item Quelle est l'interprétation physique du nombre sans dimensions $Ga= \frac{\sqrt{g L^3}}{\nu}$ (parfois appelé nombre de Galilée) ?

\item 
Des considérations aérodynamiques permettent de montrer que dans la gamme
$Ga > 10^3$ la structure de l'écoulement est indépendante de la viscosité. Estimez le nombre de Galilée pour les différents animaux volants du tableau ci-dessous.  



\item En déduire que  la vitesse d'un animal volant est proportionnel à
$M^{1/6}$.


\item Application : calculez le rapport $V/M^{1/6}$ pour les différents animaux
détaillés dans le tableau ci-dessous. Commentez les résultats
\begin{center}
\begin{tabular}{|l|l|l|l|}
\hline 
Animal  & Poids & Vitesse & $V/M^{1/6}$ \\
\hline 
Guêpe & $100 mg$ & $20 km/h$ & \\
Colibri & $2.5 g$ & $45 km/h$ & \\
Hirondelle & $17g$ & $60 km/h$ & \\
Aigle & $6kg$ & $160 km/h$ & \\ 
Airbus A300 & $150 T$ & $855 km/h$ & \\
\hline
\end{tabular}
\end{center}

\item En se basant sur l'analyse précédente, à combien estimez vous la vitesse d'une mouette de masse $400g$ ?

 \comment{
\subsubsection{Etude aérodynamique$^*$}

On souhaite retrouver la loi trouvée ci-dessus à partir des équations de l'aérodynamique. On suppose que l'animal a une aile de surface $S$ et vole à l'horizontale.

\item A partir de l'équation de la portance, exprimez la vitesse $V$ 
en fonction de $\rho$, $S$, $m$ et $g$ et $C_L$.

\item Justifiez pourquoi le coefficient $C_L$ est à peu près le même pour
tous les animaux volants.

\item Justifiez pourquoi la surface $S$ est proportionnelle à $M^{2/3}$

\item Retrouvez la loi d'échelle donnant vitesse en fonction de la masse
trouvée à la partie précédente.
}

\end{enumerate}



\comment{
\subsection{Avion et maquette$^*$}


\subsubsection{Avion en vol}

On considère un avion de transport, d'envergure 40 $m$, de longueur 45 $m$, 
de surface de voilure de 260 $m^2$. On suppose qu'il vole à 10 800 $m$ d'altitude à un nombre
de Mach $M_1 = 0.75$.  

Pour l'air on prend les caract\'eristiques suivantes :
  $r = 287.04$ et $\gamma= 1.4$ et 
  
\begin{center}
\begin{tabular}{c |cccc}
  $H$ en m     &  $T$ en   $^\circ K$    & $P$ en $kPa$   &  $\nu$ en $m^2/s$
  \\ \hline  
 10 800       & 218.08    & 23.422     &    $3.8202 \ 10^{-5}$\\
 1 800       & 276.46    & 81.494     &     $1.6869 \ 10^{-5}$
 \end{tabular}
\end{center} 
 

 \begin{enumerate}
 \item Calculer la masse volumique de l'air $\rho_1$, la vitesse du son $a_1$, 
 en déduire la vitesse de l'avion.
 \item Calculer la corde moyenne de l'aile $\ell_1$ et en déduire
 le nombre de Reynolds correspondant. Calculer le nombre de Reynolds $Re_ {L1}$
 basé sur la longueur de l'avion.
 \item Donner l'ordre de grandeur de l'épaisseur de la couche limite sur l'aile 
 à une distance $x_1= 4 \ m$ du  bord d'attaque. 
 
 \item Reprendre les questions précédentes à l'altitude de  1 800 $m$, pour le
 nombre de Mach de $M_1=0.4$.
 
 \item Conclusion
 \end{enumerate} 
 
 
 
\subsubsection{   Maquette de l'avion  au 1/10 ème}
 
   \begin{enumerate}
   \item Donner les grandeurs suivantes $\ell_2$, $L_2$, $S_2$.
    On teste la maquette dans une soufflerie reproduisant une atmosphère
   à $H=10800 \ \  m$. 
   
    \item On cherche une similitude sur le nombre de Reynolds, en déduire la vitesse
   que doit avoir l'air dans la soufflerie. Quelle est alors la valeur du nombre
   de Mach ? En tirer des conséquences sur l'aérodynamique.
   
   \item  On cherche une similitude sur le nombre de Mach, qu'elle est alors
   la valeur du nombre de Reynolds ? En tirer des conséquences sur l'aérodynamique.
   \end{enumerate}
 
 
 
 \subsubsection {   Force de portance}
 Dans les 2 cas précédents,  en déduire la valeur
 de la force de portance $L$, en sachant que le $C_L$ considéré est de $0.175$.
 Conclusions. 
 }

%
%\subsection{Analyse de similitude d'un moteur d'avion à hélice$^*$}
%
%
%On cherche des relations simples, fonctions de nombres sans dimension
%caractéristiques, qui donnent la force de traction $T$ de l'hélice, 
%le couple $Q$ exerçé par l'hélice sur l'arbre moteur
%et la puissance résistante $P_h$ de l'air sur l'hélice. 
%
%
%\begin{enumerate}
%
%\item
%Donner les dimensions et les unités de la force $T$, du diamètre de
%l'hélice $D$, de la fréquence de rotation $n$,
%de la masse volumique $\rho$, de la viscosité cinématique $\nu$, de la vitesse
%de l'avion $V_0$, et de la pression $p$
%
%
%\item En utilisant l'analyse dimensionnelle, montrer que la force de traction
%peut être mise sous la forme suivante :
%
%
%$$
%T =  \rho n^2 D^4 k_T(Re,M,J) ,
%$$
%
%où $Re = V_0 D / \nu$, $M = V_0/c$ avec $c = \sqrt{\gamma P/\rho}$, et 
%$J = V_0 /nD$.
%
%
%\item Interprétez physiquement les paramètres $Re,M$ et $J$.
%
%
%\item Trouver une relation simple du même type avec un coefficient $k_Q$ pour le
%couple $Q$.
%
%\item Ecrire la  puissance sur l'arbre de l'hélice  $P_h$  en fonction
%d'un paramètre sans dimension $C_P$.
%\item La puissance   utile pour la propulsion est 
%$P_u= T V_0$. Calculer le rendement propulsif $\eta_p$  en fonction de nombres sans dimension.
%
%\item Une hélice de diamètre $D=3.4 m$ a les caractéristiques détaillées dans
%la figure 3.
%
%%\begin{center}
%%\begin{tabular}{|c|cccc|}
%%$J$  &  1.06   & 1.19  & 1.34  &  1.44 \\
%%\hline $k_Q$ & 0.0410  & 0.04  & 0.0378  & 0.0355\\
%%\hline $\eta_p$ &  0.76  & 0.80 & 0.84  & 0.86
%%\end{tabular} 
%%\end{center}
%Déterminer la vitesse de l'avion $V_0$ qui lui permet d'absorber une puissance
%de 750 $kW$ à $n=1250  tr/mn$ et à 3660 m d'altitude ($\rho/\rho_0=0.639$) et
%donner la force de propulsion $T$. Pour le calcul numérique $n$ doit être en $tr/s$.
%
%Comparer le rendement de cette hélice à celle d'un propulseur idéal, de même
%surface, et donnant la même propulsion dans les mêmes conditions.
% 
%\end{enumerate}
%  
%   \begin{figure}
%   $$
%  \includegraphics[width = 8cm]{./courbepropulseur.eps}
%  $$
%\caption{   Figure 1 :  Abscisse : $J$, trait continu : $20 \times k_Q$, trait pointillé : $\eta_p$ }
%  \end{figure}
%

