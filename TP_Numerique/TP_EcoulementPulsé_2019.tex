\documentclass[a4,12pt]{article}
\setlength{\headsep}{-1.cm}
\setlength{\textheight}{24cm}
\setlength{\oddsidemargin}{-.5cm}
\setlength{\evensidemargin}{-.5cm}
\setlength{\textwidth}{17.cm}
\setlength{\footskip}{1.cm}
\usepackage[T1]{fontenc}
\usepackage{color}


\usepackage[utf8]{inputenc}

%\pagestyle{empty}

\usepackage{graphicx}

\begin{document}
  
\noindent
\begin{center}
\textbf{\large Licence de Mécanique, Université Paul Sabatier, 2019}\\
~\\
\textbf{\large TP num\'erique de Mécanique des Fluides}\\
~\\
\textbf{ Inspir\'e du TD: "Rythmes cardiaques"}
\end{center}

\bigskip

L'objectif de ce TP est d'approximer num\'eriquement un \'ecoulement induit entre deux plans par un gradient 
de pression oscillant.
Ce type de probl\`eme peut mod\'eliser la mani\`ere dont les battements du c{\oe}ur induisent un 
\'ecoulement puls\'e dans les vaisseaux sanguins.
En effet ces battements imposent une variation p\'eriodique du gradient de pression impos\'e 
entre deux sections transverses des vaisseaux.
L'objectif de cet exercice est de caract\'eriser ce type d'\'ecoulement
en particulier dans les limites des basses fr\'equences (organisme au repos) 
et hautes fr\'equences (activit\'e cardiaque intense).

\section{Présentation du problème et solution analytique}


Consid\'erons un mod\`ele d'\'ecoulement plan entre deux plaques
planes distantes de $2h$, g\'en\'er\'e par un gradient de pression
sinuso\"{\i}dal \`a pulsation $\omega$ fix\'ee~:
\[
\frac{1}{\rho} \frac{dp}{dx} = K \cos\omega t.
\]
\noindent
Ici $K$ correspond à l'amplitude du gradient de pression impos\'e (renormalisé par la masse volumique).
\begin{figure}[htb]
  \begin{center}
    \setlength{\unitlength}{1mm}
    \begin{picture}(60, 40)(0, 0)
      \put(0, 0){\includegraphics[width=6cm]{ecoulement_pulse}}
      \put(0, 6){$-h$}
      \put(0, 22){$+h$}
      \put(20, 20){$\partial p/\partial x = K \cos \omega t$}
    \end{picture}
  \end{center}
  \caption{\'Ecoulement puls\'e en conduite.}
  \label{fig:ecoulement_pulse}
\end{figure}

 Dans l'hypoth\`ese d'un \'ecoulement plan parall\`ele, et d'une vitesse $u$ nulle sur les 
parois inf\'erieure et sup\'erieure,
la vitesse horizontale $u(y, t)$ v\'erifie l'\'equation:
\begin{equation}
\label{par}
  \frac{\partial u}{\partial t} 
  = 
  - K \cos \omega t + \nu \frac{\partial^2 u}{\partial y^2}
  \label{eq:ns}
\end{equation}
associé aux conditions limites 
$$
u(\pm h ,t) = 0.
$$



\noindent
Sous l'hypothèse d'un {\em régime établi}, le problème admet une solution théorique qui sera notée $u_{th}(y,t)$. 
Le forçage étant de la forme $\cos \omega t $ on peut rechercher la solution sous la forme $u_{th}(y,t) = u_c(y) \cos \omega t + u_s(y) \sin(\omega t)$. 
Il est cependant plus efficace d'utiliser la méthode de la variable complexe en posant 

$$u_{th}(y, t) = Re \left \{ \underline{U}(y) e^{i \omega t} \right \}$$ 

où $\underline{U}(y) = u_c(y) - i u_s(y)$ est une fonction à valeur complexe. En remplaçant $\cos \omega t$ par $e^{i\omega t}$  dans l'\'equation~(\ref{eq:ns}), et en cherchant la solution vérifiant les conditions limites $u(\pm h,t) = 0$, on obtient finalement l'expression du profil de vitesse instationnaire suivant~:
\begin{equation}
\label{sol}
 \underline{U}(y) = \frac{iK}{\omega} \left \{ 
1 - \frac{\cosh [ ( 1+i) y \sqrt{\omega/2\nu} ]}{\cosh [ ( 1+i) h \sqrt{\omega / 2 \nu}  ]}
\right \}
\end{equation}

On ne demande pas de démontrer cette solution théorique. En revanche, on fournit un programme \textsc{Matlab} \texttt{Visualisation\_Ecoulement\_Pulse.m} 
qui permet de visualiser cet écoulement
\footnote{ Notez que dans le programme les résultats sont adimensionés en posant $h=1,\nu = 1,K=1$ si bien que $St \equiv \omega/2$.}.
 Ce programme appelle une function  \texttt{Utheo.m} calculant la solution théorique correspondant à l'équation (2), 
et effectue une boucle sur le temps pour tracer le profil de vitesse à des instants successifs.




\section{Visualisation et analyse de la solution théorique}

\begin{enumerate}

\item On définit le nombre de Strouhal du problème ainsi :
$$
St = \frac{\omega h^2}{2 \nu}
$$

Quelle est l'interprétation de ce nombre ? 
%A quels comportements peut-on s'attendre dans les limites respectives $St \ll 1$ et $St \gg 1$ ?

\item 
A l'aide de ce programme visualiser les profils de vitesse pour différentes valeurs de $St$ prises dans l'intervalle $10^{-3} \leq St <10^3$.
Repr\'esenter sch\'ematiquement sur votre compte-rendu la forme du profil de vitesse
à différents instants du cycle, dans les deux r\'egimes asymptotiques des grands et petits $St$.
Commentez physiquement les résultats. Précisez notamment à quel instant du cycle le débit dans la conduite est maximal, et expliquez physiquement.

%Montrez que le nombre d

%(on ne demande pas la démonstration de cette expression).


%\item Retrouver les solutions (\ref{sol}) \`a partir de l'\'equation (\ref{par}).


%%% Pauline : tau = h^2/nu par analyse dimensionnelle de l'équation sans second membre

%\item En adimensionnant $t$ et $y$ respectivement par $1/\omega$
%et $h$ la solution se met sous la forme :
%\begin{equation}
%\label{soladim}
 %\underline{U}^\star(y^\star) = \frac{i\gamma}{\Omega^2} \left \{
%1 - \frac{\cosh [ y^\star ( 1+i) \sqrt{\Omega} ]}{\cosh [ 
%( 1+i) \sqrt{\Omega} ]} \right \}
%\quad
%\mbox{avec} \quad
%\Omega = \frac{\omega h^2}{2\nu}, \; \gamma = \frac{Kh^3}{4\rho \nu^2}
%\end{equation}
%\noindent 
%avec $^\star$ exprimant le fait que ces variables sont adimensionn\'ees. 


\item  Montrez que le nombre de Strouhal peut aussi s'écrire sous la	 forme $St = h^2/\delta^2$ où $\delta$ est une échelle de longueur
dont vous préciserez la signification. Comment se manifeste cette échelle de longueur dans le régime $St\gg 1$ ?

\item Si l'on part d'une condition initiale donnée (par exemple $u(y,0) =0$), la solution exacte du problème est $u(y,t)= u_{th}(y,t) + u_{trans}(y,t)$
où $u_{th}(y,t) $ est la solution périodique en régime établi déterminée plus haut et $u_{trans}(y,t)$ est représente le comportement transitoire qui tend vers $0$ lorsque $t$ augmente. Donnez une estimation du temps caractéristique d'amortissement de ce transitoire. 



\end{enumerate}

\section{Calcul num\'erique}
On propose de r\'esoudre num\'eriquement cette \'equation de diffusion 
1D instationnaire forc\'ee.
Ceci permettra d'une part de comparer l'approximation numérique avec
la solution théorique en régime établi, et d'autre part d'observer
le régime transitoire précédent la convergence vers ce régime établi.

 La solution num\'erique est calcul\'ee en utilisant des m\'ethodes aux diff\'erences finies.


La discrétisation spatiale est effectuée en introduisant $N_y$ points intérieurs au domaine et définis
par $y_i = (i \Delta y -h)$ avec $i=1..N_y$ et $\Delta y = 2 h /(N_y+1)$.

La discrétisation temporelle est effectuée en posant  $t^{(n)}=n\Delta t$ avec $n=0,..,N_t$.

Enfin, on pose $u_i^{(n)} = u(y_i,t^{(n)})$.

Le sch\'ema utilis\'e sera un Euler explicite d'ordre 1 en temps, et des diff\'erences finies centr\'ees d'ordre 2 en espace.
En utilisant ce schéma l'\'equation (\ref{par}) prend la forme:
\begin{equation}
\label{disc}
\frac{u_i^{(n+1)}-u_i^{(n)}}{\Delta t}=- K \cos( \omega t^{(n)} ) + \frac{\nu}{\Delta y^2} 
\left[ u_{i+1}^{(n)}-2u_i^{(n)}+u_{i-1}^{(n)} \right] 
\end{equation}
\noindent


Remarquons que les $y_i$ d\'ecrivent les points \`a l'int\'erieur du domaine, et du fait des conditions aux limites 
de non-glissement impos\'ees sur les parois on a $u(y_0,t^{(n)})=u(y_{N_y+1},t^{(n)})=0, \ \forall n$.

Pour la résolution numérique deux stratégies sont possibles :
\begin{itemize}
\item 
On peut construire un tableau à deux dimensions \verb| U(i,n)| contenant toutes les valeurs $u_{i}^{(n)}$, et remplir les unes après les autres les colonnes du tableau correspondant aux valeurs de $u$ aux instants successifs. L'inconvénient est de nécessiter un important stockage de mémoire, inutile si l'on ne s'intéresse qu'à l'état final.
\item 
En notant que le schéma ne fait intervenir que la solution à deux pas de temps successifs, on peut se contenter d'utiliser seulement deux vecteurs colonnes 
\verb| Ufutur | et \verb| Upresent |  contenant à chaque pas de temps les valeurs à l'instant 
"présent" $t^{(n)}$ et "futur"  $t^{(n+1)}$ (c.a.d. $ U^{present} = [u_1^{(n)}; u_2^{(n)}; ... u_{N_y}^{(n)} ] $ et   $U^{futur} = [u_1^{(n)}; u_2^{(n)}; ... u_{N_y}^{(n)}$).
\end{itemize}


\begin{enumerate}
\item Montrez qu'avec cette seconde idée le calcul de la solution à chaque itération temporelle se met sous la forme :
\begin{equation}
\label{mat}
U^{futur} =A\, U^{present} +F cos(\omega t^{n}).
\end{equation}
\noindent
Où $A$ est une matrice carrée correspondant 
\`a la discr\'etisation de l'op\'erateur Laplacien et $F$ est un vecteur colonne 
repr\'esentant le terme de for\c{c}age. Pr\'eciser les dimensions des
matrices et vecteurs mis en jeu dans (\ref{mat}). 

\item Ecrire un programme effectuant l'intégration temporelle du problème,
en fonction des différents paramètres physiques et numériques,
en partant de la condition initiale $u(y,t=0) = 0$, et jusqu'à un instant final $t_f$.

Dans votre comte-rendu, vous détaillerez la stratégie de programmation utilisée.

\item A l'aide de votre programme, calculez et tracez la solution numérique obtenue 
à un instant $t_f$ suffisamment grand pour que le régime établi soit atteint (on pourra prendre par exemple $t_f = 10 T$ ; $T =\pi / \omega$,  ou adapter cette valeur selon le cas étudié).
%où $\tau$ est le temps caractéristique
%déterminé précédemment, 

Vous utiliserez les valeurs des paramètres donnés dans le tableau:
$$
\begin{array}{|c|c|c|c|c|}
\hline
h & K & \nu & \omega & N_y \\
\hline
\hspace{2cm} & \hspace{2cm} & \hspace{2cm} & \hspace{2cm} & 100 \\
\hline
\end{array}
$$
Vous déterminerez vous-même une valeur de $\Delta t$ à utiliser pour aboutir
à un bon compromis entre précision et temps de calcul.

\item Quel est le nombre de Stokes correspondant à ces paramètres ? Est-on dans l'un des régimes limites identifiés dans la partie théorique ?

%%% PAULINE : la condition de stabilité théorique est delta t / (delta y)^2 < 1/2

\item Tracez sur le même graphique la solution théorique $u_{th}(y,t_f)$.

\item Qu'observe-t-on avec des valeurs de $\Delta t$ trop grandes ? Expliquez.

\item (Question facultative) : Reprenez le problème avec une valeur $N_y$ plus élevée (par exemple $N_y=200$ ou $N_y=500$) (vous devrez éventuellement adapter le pas de temps). Calculez  l'erreur $max | u(y,t_f) - u_{th}(y,t_f)|$. Comment cette erreur varie en fonction de $N_y$ ? que peut-on en déduire sur la précision du schéma ?

\item Calculez maintenant à l'aide de votre programme la solution 
correspondant à des instant $t_f$ plus courts (de l'ordre de  $t_f = T$ ou même encore plus courts).
Comparez avec la solution théorique. A partir de ces résultats estimez le temps d'amortissement du régime transitoire et comparez avec 
l'estimation donnée à la question 4 de la partie théorique.




\item R\'e\'ecrire le sch\'ema (\ref{disc}) de mani\`ere implicite avec un schéma de Cranck-Nicholson.
(c'est-\`a-dire que la discr\'etisation du laplacien est exprim\'ee au temps $n+1/2$, défini comme 
moyenne des valeurs aux instants $n$ et $n+1$). Précisez l'ordre de ce schéma, et le traitement 
choisi pour le second membre de l'équation différentielle.

%%% PAULINE : prendre $cos ( \omega t^{n+1/2} ) $ pour être cohérent 

Ecrire ce nouveau schéma sous forme matricielle. 

\item Ecrire un nouveau programme effectuant l'intégration numérique de ce nouveau schéma.

\item Toujours pour les paramètres donnés dans le tableau et pour $t_f = 10 (2\pi/\omega)$, tracez et imprimer la solution numérique obtenue pour plusieurs valeurs de $\Delta t$. Comparez ces résultats à ceux obtenus avec le schéma explicite, et discutez en terme de précision, de stabilité, 
et de temps de calcul.

(on pourra estimer le temps de calcul des deux méthodes avec les commandes matab \verb| tic | et \verb| toc |).




\fontfamily{cmss}\fontshape{n}\selectfont
\fontfamily{cmss}\fontshape{n}\selectfont



%~\\

\noindent 

\begin{center}
{\Large PLANNING DES SEANCES (2018)}
\end{center}
\noindent


{\bf Groupe 1 :} vendredi 8 février, 08:45-11:45, salle U3-207 

{\bf Groupe 3 :} vendredi 8 février, 13:15-16:15, salle U1 205 

{\bf Groupe 2 :} vendredi 22 février, 13:15-16:15, salle U3 206

\noindent
Les étudiants ayant déjà validé des TPs l'année dernière devront se présenter lors de la première séance pour faire connaitre leurs intentions à l'enseignant.

%08/02 - 08:45-11:45 : U3-207 Groupe 1
%08/02 - 13:15-16:15 : U1 205 Groupe 3
%22/02 - 13:15-16:15 : U3 206 Groupe 2


\clearpage



~\\
%\clearpage
\begin{center}
{\Large CONSIGNES POUR LA REDACTION DES COMPTES-RENDUS}
\end{center}
\noindent
Le compte-rendu sera réalisé par groupes de deux personnes.

\noindent
Le rapport ne doit pas excéder 4 feuilles recto-verso hors courbes. 
Il devra présenter le problème physique concerné, la modélisation numérique utilisée, 
la stratégie de programmation utilisée et les résultats obtenus. Dans la présentation on  veillera a distinguer les aspects de validation numérique (précision, stabilité, etc...) des résultats physiques. 
Le rapport devra également indiquer les éventuelles difficultés rencontrées dans la mise en place ainsi que l'état d'avancement par rapport aux objectifs du TP. 
 
\noindent
 Les figures devront \^etre numérotées, citées et commentées dans le texte. Chaque figure doit obligatoirement comporter un titre clair (ou une légende située sous la figure), et les axes doivent être gradués et identifiés. (vous pouvez utiliser les commandes matlab \verb| title, xlabel|,
 \verb| ylabel | ).

Toute figure incomplète ne sera pas considérée dans la note.

Le (ou les) programme(s) écrit(s) devra(ont) être fourni(s) en annexe du rapport, ou déposé sur MOODLE comme un document séparé. 



\noindent

Une suggestion de plan pour le rapport est le suivant :
\begin{enumerate}
\item Partie théorique :

Répondre aux questions de la section 2)

\item Stratégie numérique : 

Préciser les schémas implémentés (Euler et/ou Crank-Nicholson) et la stratégie de programmation choisie
(maxi 1/2 page, complété par les programmes en annexe).

\item Validation du code numérique :

Comparez la solution calculée avec le (ou les) programme(s) avec la solution théorique, pour un instant $t_f$ suffisant pour que le régime transitoire ne soit plus présent. Commentez en terme de précision, de stabilité, et de performance (temps de calcul).

\item Etude physique du transitoire.

En utilisant Euler ou Crank-Nicholson,  Illustrez le comportement physique au cours du transitoire.


\item Conclusions
\end{enumerate}


\begin{center}
{\Large PROCEDURE DE DEPOT DES COMPTES-RENDUS}\\

Les comptes-rendus seront à rendre SUR MOODLE, au plus tard 4 semaines après 
la séance de TP (lundi 4 mars pour le groupe 1 et 3, lundi 18 mars pour le groupe 2).

Si le rapport a été rédigé par ordinateur, inutile de rendre une version papier.

Si le rapport a été rédigé à la main, vous devrez scanner (ou prendre en photo) le rapport pour le déposer sur moodle, et rendre également une version papier au cours suivant.







~
~\\


\begin{center}
{\large IL EST RAPPELE QUE LES TPS SONT OBLIGATOIRES.}
\end{center}

\end{center}






\end{enumerate}


\end{document}
