% !TEX root = TD_fluides_part1.tex


\section{Ecoulements visqueux I : Rhéologie et écoulements stationnaires parallèles}


\setcounter{subsection}{-1}

\subsection{Ecoulement de Poiseuille} 

{\em (exercice préparatoire, correction sur moodle)} 


%--------------------------------------------------------------------------------------------------
\subsection{Ecoulement de dentifrice}
%--------------------------------------------------------------------------------------------------

Le dentifrice est un exemple de "fluide \`a seuil" qui ne s'\'ecoule que si la contrainte 
de cisaillement est sup\'erieure \`a un seuil $\tau_c$. 
Consid\'erons donc un tube cylindrique de longueur $L$ rempli de dentifrice, et soumis \`a 
une diff\'erence de pression $p_e - p_s$. 
\begin{enumerate}
\item 
  D\'eterminer le profil de la contrainte de cisaillement dans le tube. 
\item 
  A partir de quelle diff\'erence de pression le dentifrice s'\'ecoule-t-il ? 
\item 
  Etablir l'expression du champ des vitesses, et repr\'esenter ce champ. 
\item 
  Etablir l'expression du d\'ebit en fonction du gradient de pression. 
\end{enumerate}



%--------------------------------------------------------------------------------------------------
\subsection{Film de peinture (Partiel 2018)}
%--------------------------------------------------------------------------------------------------

On consid\`ere un liquide de masse volumique $\rho$ uniforme qui s'écoule 
sous la forme d'un film d'\'epaisseur uniforme $h$ sur un plan inclin\'e 
d'un angle $\theta$ par rapport \`a l'horizontale. On introduit un repère $(x,y)$ où la direction 
$x$ est alignée avec le plan incliné et $y$ est la direction perpendiculaire.

%Soit $q$ le d\'ebit-volume impos\'e par unit\'e de largeur. 

Dans cet exercice on repartira de l'équation de Cauchy écrite sous la forme suivante :

$$
\rho \frac{ d \vec{u}}{d t}  =  \rho \vec{ g} - \vec{grad} p + \vec{div} ( \mytensor{\tau} )
$$

\begin{enumerate}
\item 
Sous des hypothèses que vous préciserez, montrez que l'équation de Cauchy peut
se ramener à l'équation suivante :
$$ 
 0 =  \rho g \sin \theta + \frac{\partial \tau_{xy} }{\partial y}
$$
%$$
%0 =  - \frac{\partial p}{\partial y} + \rho g \cos \theta 
%$$

\item 
Après avoir précisé la condition limite vérifiée par la contrainte en $y=h$,  établir la loi 
$\tau_{xy}(y)$ donnant la contrainte visqueuse dans le film. 

\item 
En déduire que la contrainte est maximale (en module) à la base du film et vaut :
$$ 
\tau_{xy}(y=0)  = - \rho g h \sin \theta 
$$ 
Donnez une interprétation simple de cette expression.

\item Dans le cas d'un fluide Newtonien, justifiez que la loi de vitesse $u(y)$ 
correspond à un polynôme d'ordre 2, et tracez la forme du profil $u(y)$ correspondant
(on ne demande pas la résolution mathématique complète du problème).

Dans la suite on considère le cas d'un fluide non Newtonien obéissant à la loi de Bingham:
$$
\frac{\partial u}{\partial y }  = 
\left\{ \begin{array}{ll}  0 & \quad (|\tau_{xy}| < \tau_c) \\  
\frac{1}{\mu} (\tau_{xy}-\tau_c) & \quad ( \tau_{xy} > \tau_c) \\
\frac{1}{\mu} (\tau_{xy}+\tau_c) & \quad ( \tau_{xy} < -\tau_c) 
\end{array}
\right.
$$ 

\item  Tracez la forme de la loi rhéologique ($\tau_{xy}$ en fonction de 
$\dot{\gamma} = \partial u/\partial y $) et comparez au cas d'un fluide Newtonien. Justifiez la désignation de "fluide à seuil" utilisée pour décrire ce type de fluide.

\item Montrez qu'il existe une épaisseur critique $h_c$ telle que si $h<h_c$ le film ne coule pas.
Exprimez $h_c$ en fonction de $\rho,g,\theta$ et $\tau_c$.

\item Dans le cas où $h>h_c$ tracez la forme attendue pour le profil de vitesse $u(y)$ 
(on ne demande pas une résolution mathématique complète du problème).

\item Application : 
On considère une peinture acrylique, décrite comme un fluide de Bingham avec les caractéristiques suivantes : $\tau_c = 1 Pa$ ; $\rho = 850 kg/m^3$; $\mu = 10^{-2} Pa \cdot s$. Donnez l'épaisseur maximale $h_c$ d'une couche de peinture sur une paroi verticale permettant d'éviter tout risque de coulure.

\end{enumerate}
