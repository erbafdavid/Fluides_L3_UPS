%%%%%%%%%%%%%%%%%%%%%%%%%%%%%%%%%%%%%%%%%%%%%%%%%%%%%%%%%%%%%%%%%%%%%%%%%%%%%%%%%%%%%%%%%%%%%%%%%%%
\documentclass[10pt, a4paper]{article}
%%%%%%%%%%%%%%%%%%%%%%%%%%%%%%%%%%%%%%%%%%%%%%%%%%%%%%%%%%%%%%%%%%%%%%%%%%%%%%%%%%%%%%%%%%%%%%%%%%%

%--------------------------------------------------------------------------------------------------
% Dimensions :
%--------------------------------------------------------------------------------------------------

\setlength{\textheight}{25cm}
\setlength{\textwidth}{16cm}

\setlength{\topmargin}{-20mm}
\setlength{\oddsidemargin}{0mm}
\setlength{\evensidemargin}{0mm}

% \setlength{\columnsep}{20mm}

\setlength{\fboxsep}{1mm}
\setlength{\unitlength}{1mm}

%--------------------------------------------------------------------------------------------------
% Packages :
%--------------------------------------------------------------------------------------------------

\usepackage{latexsym}
\usepackage{graphicx}
\usepackage{pifont}
\usepackage{color}
\usepackage{amsmath}
\usepackage{amssymb}

\usepackage[french]{babel}    % pour franciser le document

\usepackage[latin1]{inputenc} % pour utiliser les caracteres accentues du claviers


%--------------------------------------------------------------------------------------------------
% Divers :
%--------------------------------------------------------------------------------------------------

\definecolor{rougefonce}{rgb}{0.7, 0.2, 0.2}

\newcommand{\thickline}[2]{\linethickness{#1} \line(1, 0){#2}}
\newcommand{\mycaption}[1]{\caption{\sl #1}}
\newcommand{\myvec}[1]{\vec{#1}}

\pagestyle{empty}

\graphicspath{{Figures/}} % chemin d'acces au repertoire des figures (par ex.)

%%%%%%%%%%%%%%%%%%%%%%%%%%%%%%%%%%%%%%%%%%%%%%%%%%%%%%%%%%%%%%%%%%%%%%%%%%%%%%%%%%%%%%%%%%%%%%%%%%%
\begin{document}
%%%%%%%%%%%%%%%%%%%%%%%%%%%%%%%%%%%%%%%%%%%%%%%%%%%%%%%%%%%%%%%%%%%%%%%%%%%%%%%%%%%%%%%%%%%%%%%%%%%

\begin{center}

  \textsc{Universit� Toulouse 3 -- Paul Sabatier \hfill Ann�e universitaire 2015-2016}
  
  \textsc{M�canique des fluides \hfill L3 M�canique}
  
  \vspace{0mm}
  
  \begin{center}
    \thickline{0.4mm}{160}
    \\ \vspace{3mm}
  \textbf{\large Exercice compl�mentaire 1 : �quation de Laplace\,-Young (correction)}
    \\ %\vspace{1mm}
    \thickline{0.4mm}{160}
  \end{center}

%  \vspace{0mm}
  
\end{center}

\stepcounter{section}

\medskip

%==================================================================================================
\section{Interface � simple courbure}
%==================================================================================================

\noindent
Force de pression sur l'interface exerc�e par le fluide � l'int�rieur de la courbure : 
$d\myvec{F}_i = -p_i \, \myvec{n} \, dS$.
\\
Force de pression sur l'interface exerc�e par le fluide � l'ext�rieur de la courbure : 
$d\myvec{F}_e = p_e \, \myvec{n} \, dS$.
\\
D'o� la composante verticale des forces de pression : $dF_z = (p_e-p_i) \, dS$.

\medskip
\noindent
Composante verticale de la force de tension de surface (figs. 2b et 2c) :

$dT_z = 2 \times \gamma \times dy \times \sin (d\theta/2) \sim \gamma d\theta dy$,
car $\sin (d\theta/2) \sim d\theta / 2$ pour $|d\theta| \ll 1$.

\medskip
\noindent
Equilibre des forces verticales : 

$dF_z+dT_z = 0$, soit $(p_e-p_i) \, dS + \gamma d\theta dy = 0$,
avec $dS = dx\, dy = Rd\theta \, dy$, 

d'o� $p_e - p_i + \gamma/R = 0$,
ou encore \fbox{$p_i - p_e = \gamma/R$}

\medskip

%==================================================================================================
\section{Interface � double courbure}
%==================================================================================================

\noindent
La composante verticale des forces de pression est identique au cas pr�c�dent : $dF_z = (p_e-p_i) \, dS$.

\medskip
\noindent
Composante verticale de la force de tension de surface (figs. 3b et 3c) :

$dT_z = 2 \times \gamma \times dy \times \sin (d\theta_1/2) + 2 \times \gamma \times dx \times \sin (d\theta_2/2)
\sim \gamma d\theta_1 dy + \gamma d\theta_2 dx$,

car $\sin (d\theta_1/2) \sim d\theta_1 / 2$ et $\sin (d\theta_2/2) \sim d\theta_2 / 2$ 
pour $|d\theta_1| \ll 1$ et $|d\theta_2| \ll 1$.

\medskip
\noindent
Equilibre des forces verticales : 

$dF_z+dT_z = 0$, soit $(p_e-p_i) \, dS + \gamma ( d\theta_1 dy + d\theta_2 dx ) = 0$,

avec $dx = R_1 d\theta_1$, $dy = R_2d\theta_2$ et donc $dS = dx\, dy = R_1 R_2 d\theta_1 d\theta_2$. 

On en d�duit $p_e-p_i \, dS + \gamma ( 1/R_1 + 1/R_2 ) = 0$
ou encore \fbox{$p_i - p_e = \gamma( 1/R_1 + 1/R_2 )$}

\medskip

\noindent
\textbf{Questions subsidiaires :}

\begin{enumerate}
\item
Le cas de la courbure simple correspond aux rayons de courbure principaux $R_1=R$ (fig. 2b) et $R_2=\infty$
(fig. 2c : une portion de droite a une courbure nulle et donc un rayon de courbure infini).
On en d�duit $1/R_1 + 1/R_2 = 1/R + 0$ d'o� \fbox{$p_i - p_e = \gamma /R$}
\item
Pour une interface sph�rique (par ex. une petite bulle d'air dans l'eau, une goutelette d'eau dans l'air,
ou encore une goutelette d'huile dans l'eau), les deux rayons de courbure principaux sont identiques :
$R_1 = R_2 = R$ o� $R$ d�signe le rayon de l'interface sph�rique.
\\
On en d�duit $1/R_1 + 1/R_2 = 1/R + 1/R = 2/R$ d'o� \fbox{$p_i - p_e = 2\gamma /R$}
\item
La bulle de savon est constitu�e d'une mince film liquide de savon de forme sph�rique enserrant une volume d'air.
Il s'agit donc d'une double interface. On notera $R$ le rayon de la bulle, $p_e$ la pression de l'air � l'ext�rieur, 
$p_0$ la pression
dans le film liquide, et $p_i$ la pression de l'air dans la bulle.

L'�quation de Laplace au passage de la premi�re interface entre l'air ext�rieur et le liquide du film donne :
$p_0 - p_e = 2\gamma/R$ (cf. question pr�c�dente).

En supposant que le film est mince, le rayon de la seconde interface, entre le liquide du film et l'air � l'int�rieur de la bulle, est tr�s peu diff�rent du rayon externe $R$. 
L'�quation de Laplace � la travers�e de cette interface s'�crit alors : 
$p_i - p_0 = 2\gamma/R$.

On en d�duit le saut de pression � la travers� de la bulle : \\
$p_i - p_e = (p_i-p_0) + (p_0-p_e) = 2\gamma/R + 2\gamma/R$ d'o� \fbox{$p_i - p_e = 4\gamma/R$}.

\end{enumerate}


%%%%%%%%%%%%%%%%%%%%%%%%%%%%%%%%%%%%%%%%%%%%%%%%%%%%%%%%%%%%%%%%%%%%%%%%%%%%%%%%%%%%%%%%%%%%%%%%%%%
\end{document}
%%%%%%%%%%%%%%%%%%%%%%%%%%%%%%%%%%%%%%%%%%%%%%%%%%%%%%%%%%%%%%%%%%%%%%%%%%%%%%%%%%%%%%%%%%%%%%%%%%%

