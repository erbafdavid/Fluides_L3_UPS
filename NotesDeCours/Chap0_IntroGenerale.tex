
% !TEX root = NotesDeCours.tex



\begin{frame}

  \color{bleu}

  \begin{flushleft}
    
    \Large
   	\bf
    
    Mécanique des fluides 	
    
  \end{flushleft}
  
  \ligne{3} % remplace: \noindent \thickline{0.5mm}{150}

  \begin{flushright}

    \rm

    \textrm{David} \textsc{Fabre}
    
    \vspace{3mm}
    
    IMFT / UPS
    
    Département de Mécanique
    
david.fabre@imft.fr

  \end{flushright}


  \begin{picture}(110, 30)(10, -20)
    \put(15, -17){\includegraphics[height=42mm]{./Figures/splash.jpg}}
  \end{picture}

  \vspace{2mm}
  
  \begin{flushright}
    
    \Large
   	\bf
    
    0. Présentation et organisation du cours

  \end{flushright}


\end{frame}


%==================================================================================================
%\part{Présentation du cours}
%==================================================================================================

%--------------------------------------------------------------------------------------------------
\begin{frame}{Objectifs du cours}
%--------------------------------------------------------------------------------------------------

\small

\begin{itemize}
\item[\checkmark]
	Savoir analyser, décrire et caractériser les écoulements 
\item[\checkmark]
	Connaître les phénomènes physiques de base impliqués dans les écoulements
\item[\checkmark]
	Maîtriser la modélisation de ces mécanismes et leur mise en équation : 
\item[] $\rightarrow$ Equation de Navier-Stokes
\item[\checkmark]
	Savoir identifier les mécanismes dominants et ceux qui sont négligeables
\item[]
	$\rightarrow$ définition et exploitation des nombres sans dimension.
\item[\checkmark]
	Savoir simplifier les modèles en conséquence
\item[\checkmark]
	Connaitre les principales techniques de résolution mathématique de ces problèmes
\item[]
	$\rightarrow$ Méthodes locales (résolution exacte de l'équations de NS dans des cas simples) et intégrales (utilisation d'équations-bilan). 
\end{itemize}

\bigskip

\hfill Remarque : liste non exhaustive\ldots

\vspace{20mm}

\end{frame}




%--------------------------------------------------------------------------------------------------
\begin{frame}{Description du module}
%--------------------------------------------------------------------------------------------------

\small


\textbf{Format :} \medskip

\begin{itemize}
\item[\checkmark]
	12 Cours (12x2h) + 12 TD (12x2h) + 3 TPs expérimentaux (3x3h) + 1 TP numérique (3h).
\item[\checkmark]
	Structuration du cours : 1 chapitre = 1 thème par semaine 
	\mytabbing{Structuration du cours :} avec 1 séance de TD associé (2h, la semaine d'après).
	\mytabbing{Structuration du cours :} (NB :  2 cours la première semaine !)
	
\item[\checkmark]
	Entre cours et TD : \textcolor{rouge}{exercice complémentaire} sur Moodle (travail personnel)
%	\mytabbing{Entre cours et TD : } Questionnaire Pédagogique Hebdomadaire (sur moodle)
\item[\checkmark]
	TP expérimentaux et numériques : obligatoires, cf. informations sur le tableau d'affichage
\end{itemize}

\pause
\medskip

\textbf{Intervenants :} \medskip

\begin{itemize}
\item
	Cours : David FABRE (david.fabre@imft.fr)
\item
	TD : Mokhtar ZAGZOULE, Pierre BRANCHER
\item
	TP expérimentaux : Frédéric MOULIN
\item
	TP numériques : David FABRE
	
\item ( Enseignants précédents : F. MOULIN, P. LAURENS, S. SAINTLOS, F. CHARRU... )	
\end{itemize}

\pause
\medskip

\textbf{Evaluation :} \medskip

\begin{enumerate}
\item
	Première session : TP 25\% (num 10\%, expé 15 \%), CC 30\% (exam partiel 2h), CT 45 \% (exam final 3h)  
\item 
	Seconde session : report de la note de TP (20\%), examen terminal 2 (80\%)
\end{enumerate}

\vspace{5mm}

\end{frame}



%--------------------------------------------------------------------------------------------------
\begin{frame}{"Philosophie"}
%--------------------------------------------------------------------------------------------------

\small

Mécanique des fluides = discipline scientifique dont la maîtrise passe par la pratique régulière et
\mytabbing{Mécanique des fluides =} répétée des analyses et des techniques de modélisation et de résolution 
\mytabbing{Mécanique des fluides =} (exercices de TD, développements théoriques et démonstrations du cours)

\bigskip

\qquad $\rightarrow$ \textcolor{rouge}{travail personnel !} \quad (rappel : 1h de présentiel = 1h de travail perso)

\medskip
\qquad $\Rightarrow$ sur les 15 semaines du semestre = en moyenne 3h/semaine minimum (révisions incluses)

\vspace{5mm}
\pause

\textbf{Méthodologie}

\medskip
Entre le cours et le TD : \hfill (NB : pas de rappel de cours en TD\ldots)
\begin{enumerate}
\item relire le cours, 
\item refaire les démonstrations,
\item refaire les exercices traités en cours,
\item travailler l'exercice complémentaire de la semaine.
%\item répondre au questionnaire pédagogique hebdomadaire sur moodle !
\end{enumerate}

\vspace{5mm}
\pause

\textsl{Important} : rien n'est complètement trivial, il faut \textcolor{rouge}{se} poser des questions 
$\rightarrow$ \textcolor{rouge}{posez des questions !}

\medskip
Présupposé : il n'existe (presque) pas de question stupide en mécanique des fluides\ldots

\vspace{10mm}

\end{frame}

%--------------------------------------------------------------------------------------------------
\begin{frame}{Informations}
%--------------------------------------------------------------------------------------------------

\small

%Quelques informations disponibles sur le tableau d'affichage du L3

\medskip

%Plus d'informations sur la page Moodle du cours : %\quad \texttt{\color{rouge} http://moodle.ups-tlse} 
%{\tiny \url https://moodle.univ-tlse3.fr/course/view.php?id=1025}

%\medskip

%\qquad $\rightarrow$ puis taper dans \texttt{Recherche} : mécanique des fluides

%\medskip

Espace Moodle du cours 

{\tiny \url https://moodle.univ-tlse3.fr/course/view.php?id=1025}


\pause

\bigskip

Plusieurs documents pédagogiques ou administratifs mis en ligne :

\begin{itemize}
\item
	Résumés de cours
\item
	Compléments de cours (Formulaire,...)
\item
	Enoncés de TD
\item
	\textcolor{rouge}{Exercices complémentaires} (énoncés et corrigés)
%\item
%	\textcolor{rouge}{Questionnaire Pédagogique Hebdomadaire}
\item
	Enoncés et corrections d'autres exercices et problèmes 
\item
	Programmes informatiques
\item 
	Examens et corrigés 
\item 
	$[\ldots]$
\end{itemize}

\vspace{5mm}

\end{frame}


%--------------------------------------------------------------------------------------------------
\begin{frame}{Pré-requis et références bibliographiques}
%--------------------------------------------------------------------------------------------------

\small
\textbf{Pré-requis :} \smallskip

\begin{itemize}
\item
	Cours de \textcolor{rouge}{Mécanique des milieux continus} (MMC) du premier semestre
\item
	Cours de Thermodynamique du premier semestre
\item
	Notions de mécanique du point, des solides rigides et des systèmes.

	
	
\item
	Outils mathématiques : analyse, algèbre linéaire, géométrie différentielle, 
	\mytabbing{Outils mathématiques :} équations différentielles, intégrales multiples, \ldots
	
	\item 	
	Premières notions de mécanique des fluides ; Cours L2 (M. Marcoux).
	
	Nombreux documents sur moodle :  
	
	{\tiny \url{https://moodle.univ-tlse3.fr/course/view.php?id=1797}}

	
\end{itemize}



\pause

\medskip
\textbf{Modules "compagnons" du second semestre :} \smallskip

\begin{itemize}
\item
	Transferts thermiques 
\item
	Mécanique des solides 
\end{itemize}

\pause

\medskip
\textbf{Références :} \smallskip

\begin{itemize}
\item[] \hspace{-5mm}
	Guyon, Hulin \& Petit : \textsl{Hydrodynamique physique}.
		CNRS éditions, 2001.
%\item[] \hspace{-5mm}
%	Guyon, Hulin \& Petit : 
%		\textsl{Ce que disent les fluides}.	Belin, 2005. 
\item[] \hspace{-5mm}
	Chassaing : \textsl{Mécanique des fluides : éléments d'un premier parcours}.
		Editions Cépaduès, 2000.
\item[] \hspace{-5mm}
	Candel : \textsl{Mécanique des fluides}. 
		Dunod, 2005 (3e édition).
\item[] \hspace{-5mm}
	Darrozès \& François : \textsl{ Mécanique des fluides}.
		Editions de l'ENSTA, 1998.		
		
%\item[] \hspace{-5mm}
%	Ryhming : \textsl{Dynamique des fluides}.
%		Presses Polytechniques et Universitaires Romandes, 2004.
%\item[] \hspace{-5mm}
%	Acheson : \textsl{Elementary Fluid Dynamics}. 
%		Oxford University Press, 1990.
\end{itemize}

\end{frame}

%--------------------------------------------------------------------------------------------------
\begin{frame}{Plan du cours}
%--------------------------------------------------------------------------------------------------

\small

\begin{enumerate}
\item
  Introduction -- Analyse dimensionnelle
\item
  Hydrostatique -- Forces dans les fluides au repos
  \item
  Cinématique -- Description du mouvement d'un fluide.
\item 
  Viscosité -- Forces dans les fluides en mouvement
\item
  Equations de la mécanique des fluides -- régimes d'écoulement
\item
  Ecoulements visqueux
\smallskip
\item[] \qquad $\rightarrow$ partiel
\smallskip
\item
  Ecoulements inertiels
\item
  Ecoulements potentiels
\item
  Ecoulements en conduite
\item
  Acoustique
\item
  Ecoulements compressibles
\item
  Ondes de choc
\smallskip
\item[] \qquad $\rightarrow$ examen terminal
\end{enumerate}

\vspace{5mm}

\end{frame}


