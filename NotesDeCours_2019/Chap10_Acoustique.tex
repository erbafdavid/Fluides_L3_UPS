
% !TEX root = NotesDeCours.tex


\part{Chapitre 10 : Acoustique}

% ================================================================================================ 
% Page de titre :
% ================================================================================================

\begin{frame}

  \color{bleu}

  \begin{flushleft}
    
    \Large
   	\bf
    
    Mécanique des fluides 

  \end{flushleft}
  
  \ligne{3} % remplace: \noindent \thickline{0.5mm}{150}

  \begin{flushright}

    \rm

    \textrm{David} \textsc{Fabre}
    
    \vspace{3mm}
    
    IMFT / UPS
    
    Département de Mécanique
    
%    brancher@imft.fr

  \end{flushright}

\begin{picture}(110, 40)(0, 0)
  \put( 0,  0){\includegraphics[height=47mm]{helmholtz.jpg}}
  \put(34,  0){\includegraphics[height=47mm]{resonator_crop.jpg}}
  \put(67, 8.5){\color{gris} \small \rm 
  	\begin{minipage}{38mm}
  		Hermann Ludwig Ferdinand \\ von Helmholtz (1821--1894)
		\\ tenant dans ses mains \\ le résonateur
		acoustique \\ qui porte son nom, \\ photographié ci-contre.
	\end{minipage}}
\end{picture}

  \vspace{5mm}
  
  \begin{flushright}
    
    \Large
   	\bf
    
    %\sout{11.} 
    \quad  10. Acoustique

  \end{flushright}

  \vspace{7mm}

\end{frame}

%%%%%%%%%%%%%%%%%%%%%%%%%%%%%%%%%%%%%%%%%%%%%%%%%%%%%%%%%%%%%%%%%%%%%%%%%%%%%%%%%%%%%%%%%%
% Sommaire :
%%%%%%%%%%%%%%%%%%%%%%%%%%%%%%%%%%%%%%%%%%%%%%%%%%%%%%%%%%%%%%%%%%%%%%%%%%%%%%%%%%%%%%%%%%

\begin{frame}{Sommaire}

\small
  
\hspace*{2mm}
\begin{tabular}{cc}
		%&
  		\begin{minipage}{62mm}
  			\tableofcontents[firstsection=-8]
      \vspace{15mm}
  		\end{minipage}
  		&   
  		\begin{minipage}{60cm}
		  \vspace*{-5mm}  
  			%\includegraphics[width=40mm]{vagues.jpg} 
  		\end{minipage}
  	\end{tabular}

\vspace{0mm}

\end{frame}

%%%%%%%%%%%%%%%%%%%%%%%%%%%%%%%%%%%%%%%%%%%%%%%%%%%%%%%%%%%%%%%%%%%%%%%%%%%%%%%%%%%%%%%%%%
\section{\bfseries Acoustique}
%%%%%%%%%%%%%%%%%%%%%%%%%%%%%%%%%%%%%%%%%%%%%%%%%%%%%%%%%%%%%%%%%%%%%%%%%%%%%%%%%%%%%%%%%%
%==========================================================================================
\subsection{Equations de l'acoustique}
%=========================================================================================

%-----------------------------------------------------------------------------------------
\subsubsection{Cadre de la modélisation}
%-----------------------------------------------------------------------------------------
\begin{frame}{Cadre de la modélisation}
%-----------------------------------------------------------------------------------------

\small

\pause

\textbf{Cadre :} \medskip

état de base : fluide au repos
\[
	\myvec{u}_0 = \myvec{0}
\]
et grandeurs thermodynamiques uniformes (pas de variation spatiale) 
\[
	p=p_0, \quad T=T_0, \quad \rho = \rho_0
\]

Gravité négligée (ou alors on pose $\hat p_0 = p_0+\rho_0 g z$)

\bigskip \pause

\textbf{Objectif :} \medskip

déterminer l'évolution de perturbations de cet état de base :
\[
	\myvec{u} = \myvec{u}_0 + \myvec{u}\myprime(\myvec{x}, t), \quad
  p = p_0 + p\myprime(\myvec{x}, t), \quad
  T = T_0 + T\myprime(\myvec{x}, t), \quad
  \rho = \rho_0 + \rho\myprime(\myvec{x}, t)
\]

\bigskip \pause

\textbf{Hypothèses :} \medskip

on supposera ces perturbations "petites" (ou infinitésimales), d'ordre $\varepsilon \ll 1$
\[
	\frac{p\myprime}{p_0}, \; \frac{T\myprime}{T_0}, \; \frac{\rho\myprime}{\rho_0}, 
	\; 
	\frac{\myvec{u}\myprime}{c_0}
	\sim \varepsilon \ll 1	
\]

Remarque : la seule échelle de vitesse est $c_0 = 1/\sqrt{\rho_0 \chi_s}$.

\smallskip 
Echelle de temps associée : $\tau_{ac} =  L/c_0$ 


\vspace{10mm}

\end{frame}

%-----------------------------------------------------------------------------------------
\subsubsection{Equation de la quantité de mouvement}
%-----------------------------------------------------------------------------------------
\begin{frame}{Conservation de la quantité de mouvement : Analyse dimensionnelle}
%-----------------------------------------------------------------------------------------

\small

Equation locale de bilan de quantité de mouvement (cf. chapitre 5) :
\begin{equation}
\begin{array}{ccccccc}
\underbrace{\rho \dpdt{\myvec{u}}}_{\color{green}{[I]}} 
&+& 
\underbrace{\rho (\mytensor{grad} \myvec{u} ) \cdot \myvec{u}}_{\color{green}{[A]}} 
 &=&
\underbrace{ - \gradient p' }_{\color{green}{[P]}} 
 &+& 
 \underbrace{ \mu \left( \Delta \myvec{u} - \gradient (div(\myvec{u})/3)  \right)}_{\color{green}{[V]}} 
  \\
  \color{red}{\frac{\rho_0 [u']}{\tau_{ac}}}
  &&
   \color{red}{\frac{\rho_0 [u']^2}{L}}
&&
   \color{red}{\frac{[p']}{L}}
   &&
    \color{red}{\frac{\mu [u'] }{L^2}}
   \end{array}
\end{equation}

\pause

Régime Acoustique : le terme dominant est \textcolor{green}{[I]}.


\begin{itemize}
\item $\frac{[A]}{[I]}  \ll 1 \Longleftrightarrow \frac{[u']}{c_0} = Ma \ll 1$

\item $\frac{[V]}{[I]}  \ll 1 \Longleftrightarrow \frac{c_0 L}{\nu } = Ma / Re \equiv Kn \ll 1$

\item Le PMD permet de déterminer la jauge de pression $[p']$  : 
$\frac{[P]}{[I]}  \approx 1 \Longleftrightarrow \quad [p'] = \rho_0 [u'] c_0 $.
\end{itemize}


Sous ces hypothèses l'équation-bilan de QdM se simplifie alors en :

$$
\color{vert}
\rho_0 \dpdt{\myvec{u}\myprime} = - \gradient p \myprime
$$

\end{frame}



%-----------------------------------------------------------------------------------------
\subsubsection{Equation de la masse}
%-----------------------------------------------------------------------------------------
\begin{frame}{Conservation de la masse : analyse dimensionnelle}
%-----------------------------------------------------------------------------------------

\small

Equation locale de bilan de masse (cf. chapitre 5) :
\begin{equation}
\begin{array}{cccccc}
   \underbrace{\dpdt{\rho}}_{\textcolor{green}{[I]}} 
   &+&
\underbrace{ \myvec{u} \cdot \gradient(\rho)}_{\textcolor{green}{[A]}}  
 &+& 
   \underbrace{\rho  \divergence (\myvec{u})}_{\textcolor{green}{[D]}}  
   &= 0 \\
    \color{red}{\frac{[\rho']}{\tau_{ac}}}
  &&
   \color{red}{\frac{[\rho'] [u']}{L}}
&&
   \color{red}{\frac{\rho_0[u']}{L}}
\end{array}   
\end{equation}

\pause \bigskip


Régime Acoustique : le terme dominant est \textcolor{green}{[I]}.


\begin{itemize}
\item $\frac{[A]}{[I]}  \ll 1 \Longleftrightarrow \frac{[u']}{c_0} = Ma \ll 1$

\item 
PMD : $\frac{[D]}{[I]} \approx 1$ 
$\Longrightarrow$ $\frac{[\rho']}{\rho_0}  = \frac{[u']}{c_0}$.
\end{itemize}



\bigskip \pause
L'équation linéarisée en perturbation s'écrit alors :
\begin{equation}
	\color{vert}
  \dpdt{\rho\myprime} + \rho_0 \divergence \myvec{u}\myprime = 0
\end{equation}

\vspace{5mm}

\end{frame}

%-----------------------------------------------------------------------------------------
\subsubsection{Compressibilité et vitesse du son}
%-----------------------------------------------------------------------------------------
\begin{frame}{Compressibilité : évolution thermodynamique}
%-----------------------------------------------------------------------------------------

\small

Equation-bilan de l'énergie interne (cf. chap 5  et annexe A.5)

\begin{equation}
		\rho \ddt{e} 
		= \mytensor{\tau} : \mytensor{grad}(\vec{u}) + 
		p \divergence( \vec{u})
		 - \divergence(\vec{q})
		%\label{eq:bilan_local_lagrangien_qdm}
\end{equation}

Les termes de dissipation visqueuse et de conduction thermique ($\vec{q} = - \kappa \gradient T$) sont négligeables sous les conditions suivantes: 

-- $ \frac{\mu [u']^2}{L^2} \ll \frac{\rho_0 [e']}{\tau_{ac}} 
\quad \Longleftrightarrow \quad Re / Ma \gg 1$

-- $\frac{\kappa [T'] }{L^2} \ll \frac{\rho_0 [e']}{\tau_{ac}} \quad \Longleftrightarrow \quad Pe/Ma \gg 1$

(Nb de Péclet $Pe = \frac{[u'] L \rho c_p}{\kappa}$)

En utilisant l'eq. de la masse, l''eq. de l'énergie interne s'écrit alors :

$$
 \ddt{e}  =- p \ddt{}\left(\frac{1}{\rho}\right) 
$$

dans laquelle on reconnait :
$
T \ddt{s} = \ddt{e}  + p \ddt{}\left(\frac{1}{\rho}\right) =  0. 
$
 
 L'évolution est donc isentropique.
 
On peut alors relier les variations de pression et de masse volumique par:

\[
 \frac{p'}{\rho'} \approx  \left.\frac{\partial p}{\partial \rho}\right|_{\! s}  = \frac{1}{\rho \chi_s} = c_0^2.
\]


\medskip \pause
\textbf{Rappel :}  $\color{red} \kappa_s = 1/\gamma p_0$ pour un gaz parfait.

On en déduit $c_0^2 = \frac{\gamma p_0}{\rho_0} = \gamma r T_0 $


\vspace{0mm}

\end{frame}


%-----------------------------------------------------------------------------------------
%\subsubsection{Vitesse du son}
%-----------------------------------------------------------------------------------------
\begin{frame}{Vitesse du son}
%-----------------------------------------------------------------------------------------

\small

%Les prédictions théoriques précédentes font apparaître la vitesse de propagation des ondes
%acoustiques, appelée vitesse du son, dont l'expression générale est donnée par

Rappels : les propriétés thermodynamiques permettent de définir une échelle de vitesse, appelée vitesse du son, définie par

\[
	\color{vert}
	c_0 = \dfrac{1}{\sqrt{\rho_0 \kappa_s}}
\]	
où $\rho_0$ désigne la masse volumique ambiante du fluide, supposée uniforme, 
et $\kappa_s$ correspond \\ au coefficient de compressibilité isentropique du fluide
dans les conditions thermodynamiques ambiantes.

\bigskip \pause

\textbf{Dans l'air :}  \medskip

en supposant que l'air est un gaz parfait obéissant à la loi de Boyle--Mariotte 
$p_0 = \rho_0 r T_0$, \\ où $r = 287$ J/kg/K désigne la constante spécifique de l'air
\[
	\kappa_s = \frac{1}{\gamma p_0} 
	\quad \Rightarrow \quad 
	\color{red} c = \sqrt{\dfrac{\gamma p_0}{\rho_0}} = \sqrt{\gamma r T_0}
\]
soit, à $T_0 = 20 ^o  C  = 293 K$, {\color{vert} $c \sim 340$ m/s}

\bigskip \pause

\textbf{Dans l'eau :} \medskip

$\kappa_s \sim 5 \times 10^{-10}$ Pa$^{-1}$ et $\rho_0 = 10^3$ kg/m$^3$ : 
{\color{vert} $c = 1/\sqrt{\rho_0 \kappa_s} \sim 1400$ m/s}

\vspace{0mm}

\end{frame}


%-----------------------------------------------------------------------------------------
\subsubsection{Equation de Helmholtz}
%-----------------------------------------------------------------------------------------
\begin{frame}{Equation de Helmholtz}
%-----------------------------------------------------------------------------------------

\small



\medskip

\pause
Les équations du modèle mis en place précédemment 
s'écrivent donc, pour les perturbations en vitesse $\myvec{u}'$, 
en pression $p'$ et masse volumique $\rho'$ :
\begin{eqnarray}
	\dpdt{\rho'} + \rho_0\divergence \myvec{u}' & = & 0
	\\
	\rho_0 \dpdt{\myvec{u}'} + \gradient p' & = & \myvec{0}
	\\
	\dpdt{p'} & = & c_0^2 \, \dpdt{\rho'} 
\end{eqnarray}

\medskip

\pause
Par combinaison linéaire de ces trois équations, on montre que les perturbations
vérifient l'équation de l'acoustique linéaire
\begin{equation}
	\color{vert}
	\mbox{\color{gris} [Démonstration] \; $\longrightarrow$ \;}
	\ddpdt{p'} - c_0^2 \Delta p' = 0 \qquad \mbox{(équation de HELMHOLTZ)}
\end{equation}

\vspace{0mm}

\end{frame}

%==========================================================================================
\subsection{Ondes planes}
%=========================================================================================

%-----------------------------------------------------------------------------------------
\subsubsection{Solution générale}
%-----------------------------------------------------------------------------------------
\begin{frame}{Ondes planes : solutions}
%-----------------------------------------------------------------------------------------

\small
On peut rechercher des solutions générales de l'équation de Helmholtz sous la forme d'ondes, \\
appelées ondes acoustiques.
On se restreint ici à l'étude des ondes \textcolor{vert}{planes}, solutions de la forme
\[ \color{vert}
	\myvec{u} = u'(x, t) \, \vec{e}_x, \; p'(x, t).
\]
Dans ce cas l'équation de Helmholtz s'écrit sous la forme suivante (également appelée Equation de d'Alembert) :
\begin{equation}
	\color{vert}
		\ddpdt{p'} - c_0^2 \ddpdx{ p'} = 0 
\end{equation}
\pause
On remarque que cette équation peut se factoriser sous la forme :
\[
	\left (\dpdt{} + c_0\dpdx{}\right ) \left (\dpdt{} - c_0\dpdx{} \right) p'
=
	0
\]
\pause

Ou encore 
\[
- c_0^2 \frac{\partial }{\partial x_+} \frac{\partial }{\partial x_-} p' = 0
\quad \mbox{ avec le changement de variable } x_+ = x-ct \mbox{ et } x_- = x+ct. \quad 
\]



\textcolor{gris}{\small (En effet 
$\frac{\partial}{\partial x_+} = \frac{dx}{dx_+} \frac{\partial}{\partial x} + \frac{dt}{dx_+} 
\frac{\partial}{\partial t}  = 
\frac{\partial}{\partial x} - \frac{1}{c_0} \frac{\partial}{\partial t}$; de même pour $ \frac{\partial}{\partial x_-}$ ).}

\smallskip
On en déduit qu'une solution générale peut se mettre sous la forme :


\[ \color{red}
	p'(x, t) =   f(x-c_0 t) + g(x+c_0 t)
\]

On montre alors que la vitesse associée est donnée par :
\[ \color{red}
	u'(x, t) = (\rho_0 c_0)^{-1} \left[  f(x-c_0 t) - g(x+c_0 t) \right]
\]


Interprétation : ces deux termes correspondent à des ondes planes se propageant en direction positive et négative.



\vspace{0mm}



\end{frame}

\subsubsection{Réflexions sur une extrémité ouverte ou fermée}
%-----------------------------------------------------------------------------------------
\begin{frame}{Ondes planes : réflexion sur une extrémité fermée ou ouverte}
%-----------------------------------------------------------------------------------------

\small

Etudions la propagation d'une onde plane dans un tube de section
constante occupant la région ($x<0$).

\[ \color{red}
	p'(x, t) =   f(x-c_0 t) + g(x+c_0 t)
\]
\[ \color{red}
	u'(x, t) = (\rho_0 c_0)^{-1} \left[  f(x-c_0 t) - g(x+c_0 t) \right]
\]

Dans ce cas le terme $f$ correspond à une {\em onde incidente} provenant du coté $x<0$ et se propageant dans la direction positive. Le terme $g$ correspond à l'{\em onde réfléchie} générée à la position $x=0$ et se propageant dans la direction négative.

\begin{itemize}

\item Si le tuyau est fermé à son extrémité $x=0$, la condition $u'(x=0,t)$ conduit à la conclusion que {\color{red} L'onde se réfléchit en gardant le même signe}.

\textcolor{gris}{Démo : $u'(x=0,t) = (\rho_0 c_0)^{-1} [f(-c_0t) - g(+c_0t) ] = 0.$ Donc $g(s) = +f(-s)$.} 

\item Si le tuyau est {\em idéalement ouvert} à son extrémité, la condition $p'(x=0,t)=0$ conduit à 
la conclusion que {\color{red} L'onde se réfléchit en changeant de signe signe}.

\textcolor{gris}{Démo : $p'(x=0,t) =  [f(-c_0t) + g(+c_0t) ] = 0.$ Donc $g(s) = -f(-s)$.} 

\item { \color{gris} Remarque : dans le cas d'un tuyau ouvert, une modélisation plus fine aboutit à $p'(x=\Delta ,t) = 0$, où $\Delta$ est une "correction de longueur" 
qui vaut approximativement $\Delta \approx 0.81 D$ où $D$ est le diamètre du tuyau.
L'onde se réfléchit donc (en changeant de signe) à la position $x=\Delta$.}

\end{itemize}


\end{frame}



%-----------------------------------------------------------------------------------------
\subsubsection{Ondes progressives monochromatiques}
%-----------------------------------------------------------------------------------------
\begin{frame}[fragile]{Ondes progressives monochromatiques}
%-----------------------------------------------------------------------------------------

\small

Def. On appelle {\em Onde plane progressive monochromatique} une solution de la forme suivante :

$
	p'(x, t) = Re (A \, e^{i(kx-\omega t)} ) \equiv |A| \cos ( kx -\omega t + \varphi_A)
$

%où $A = |A| e^{i \phi_A}$ est une amplitude éventuellement complexe, 
%$k$ le nombre d'onde (en $rad/m$) et $\omega$ la pulsation ($rad/m$).
%\begin{itemize}
%\item
%$k$ est relié à la longueur d'onde par $k = 2 \pi / \lambda$
%\item
%$\omega$ est relié à la période $T$ par $\omega = 2 \pi /T$ et à la fréquence $f$ (en cycle/s) par 
%$\omega = 2 \pi f$.
%\item $k$ et $\omega$ sont reliés entre eux par la {\em relation de dispersion} $ \omega/k = c_0$.

%\item {\color{grey} Rem : $\omega/k$ ne dépend pas de la fréquence $\omega$, les ondes sonores sont donc {\em non dispersives}}.
%\end{itemize}

 \smallskip
 
 
\begin{picture}(117, 33)(-3, 0)
	\put(0, 0){\includegraphics[width=60mm]{mode_normal.png}}
	\put(27, 27){$\lambda = 2\pi/k$}
	\put(8, 18){$|A|$}
	\put(33, 8){$|A|$}
	\put(36.5, 15){$c$}
	\put(62, 15){%
		\begin{minipage}{55mm} 
			\begin{itemize}
			\item
				$A = |A| e^{i \varphi_A}$ : amplitude (en pression)
			\item
				$k = 2\pi/\lambda$ : nombre d'onde (en rad/m)
			\item
				$\lambda = 2\pi/k$ : longueur d'onde
			\item
				$\omega = 2 \pi /T$ : pulsation (en rad/s)
			\item 
				$f = 1/T = \omega/2 \pi$ : fréquence (en Hz)		
			\item
				$T = 2 \pi/\omega $ : période (en s)	
			\item
				$c = \omega/k$ : vitesse de phase (ou célérité)
			\end{itemize}
		\end{minipage}}
\end{picture}


\pause 

\smallskip

{\scriptsize 
Illustration multimédia :
\begin{verbatim}
http://www.animations.physics.unsw.edu.au/jw/travelling_sine_wave.htm
\end{verbatim}
}


\medskip

En injectant cette forme de solution dans l'eq. de Helmholtz, on obtient directement : $c = c_0$. 


La célérité des ondes acoustiques est donc la même pour toutes les longueurs d'ondes (et toutes les fréquences).


\medskip

Les ondes sont acoustiques sont dites \textcolor{vert}{non dispersives} : 
les sons graves se propagent donc à la même vitesse que les sons aigus\ldots

\medskip

\textcolor{gris}{
Remarque  : la situation est différente pour les ondes de surface (vagues) 
qui sont dispersives, les ondes de grande longueur d'onde se propagent plus vite que les ondes de petite longueur d'onde (cf programme M1).
}



%\vspace{39mm}

\end{frame}

\subsubsection{Ondes stationnaires}
\begin{frame}[fragile]{Solution d'onde stationnaire harmonique}

En superposant deux ondes planes de même amplitude $A$ et de direction opposées, on obtient une solution appelée {\em onde stationnaire :}
$$
p'(x,t) = A \left( \cos (kx - \omega t) + \cos ( kx + \omega t) \right) = 2 A \cos k x \cos \omega t
$$
$$
u'(x,t) = (\rho_0 c_0)^{-1} A \left( \cos (kx - \omega t) - \cos ( kx + \omega t)  \right) = 2 \rho_0 c_0 A 
\sin k x \sin \omega t
$$

Cette situation se rencontre en particulier dans les instruments de musique (cf. TD).

\medskip

{\scriptsize Illustration multimédia :
\begin{verbatim}
http://newt.phys.unsw.edu.au/jw/strings.html
\end{verbatim}
}

\vspace{39mm}

\end{frame}

%==========================================================================================
\subsection{Energie et intensité acoustiques}
%=========================================================================================

%-----------------------------------------------------------------------------------------
\subsubsection{Equation de l'énergie cinétique}
%-----------------------------------------------------------------------------------------
\begin{frame}{Equation de l'énergie cinétique}
%-----------------------------------------------------------------------------------------

\small


Considérons a nouveau le bilan local d'énergie cinétique (cf. annexe A.3)



\begin{equation}
		\rho \ddt{} \frac{ |\vec{u}|^2}{2} 
		=   - \gradient p \cdot {\vec u} + \vec{u} \cdot \divergence ( \mytensor{\tau} )
		%\label{eq:bilan_local_lagrangien_qdm}
\end{equation}

Sous les hypothèses de l'acoustique linéaire, celui-ci peut s'écrire :

\begin{equation}
		 \dpdt{} \left( \rho_0 \frac{u'^2}{2} \right)
		 = - \divergence( p' {\vec u}' ) + p' \divergence( {\vec u}' )  
		%\label{eq:bilan_local_lagrangien_qdm}
\end{equation}


\pause

\begin{itemize}
\item
	Le premier terme correspond aux taux de variation de l'énergie cinétique volumique $e_c$.
\item
	Le deuxième terme s'interprète comme la {\em puissance extérieure des forces de pression} par unité de volume
\item
 	Le troisième correspond à {\em puissance intérieure des efforts des forces de pression} par unité de volume.

	En utilisant l'équation de la masse ce terme peut s'écrire	
	\[
		p' \divergence ( \myvec{u}') = - \frac{p'}{\rho_0} \dpdt{\rho'}  = -\frac{1}{ \rho_0 c_0^2} p'  \dpdt{p'}  = -\dpdt{e_p}
	\]
	où $e_p =  \frac{p'^2}{2 \rho_0 c_0^2}$ désigne une énergie potentielle associée à la pression acoustique.
\end{itemize}

\pause
%L'équation de l'énergie cinétique a donc pour expression
%\[
%	\color{vert}
%	\dpdt{} \left(e_c+e_p \right) = -\dpdx{}(p'u')  \]

\vspace{0mm}

\end{frame}


%-----------------------------------------------------------------------------------------
\subsubsection{Equation de l'énergie acoustique}
%-----------------------------------------------------------------------------------------
\begin{frame}{Equation de l'énergie acoustique}
%-----------------------------------------------------------------------------------------

\small

En introduisant l'\textcolor{vert}{énergie acoustique} par unité de volume
\[
	\color{vert}
	e_a = e_c + e_p = \frac{1}{2} \rho_0 u'^2 + \frac{1}{2\rho_0 c_0^2} p'^2
\]
\pause
et le vecteur \textcolor{blue}{intensité acoustique} 
\[
	\color{blue}
	\vec{I} = p' \vec{u}'
\]
\pause
l'équation pour l'énergie cinétique  s'écrit alors
\[
	\color{red}
	\dpdt{e_a} = - \divergence (\vec{I})	
\]

\pause

\bigskip

\textbf{Cas d'une onde plane :} \medskip

Sachant que pour une onde plane $u' =  \pm \frac{p'}{\rho_0 c_0}$, alors $e_c =  e_p$ (équipartition de l'énergie), et donc
\[
	\color{red}
	e_a =  \rho_0 u'^2 = \frac{p'^2}{\rho c_0^2}
\]

D'autre part pour une onde plane dans la direction $x$ on a :
\[
\vec{I} = p' \vec{u}' = \pm \frac{p'^2}{\rho_0 c_0} \vec{e}_x = \pm e_a c_0 \vec e_x
\]

Interprétation : dans une onde plane, l'énergie se propage à la vitesse $c_0$

\vspace{0mm}

\end{frame}


%-----------------------------------------------------------------------------------------
\subsection{Réflexion et transmission sur une discontinuité}
%-----------------------------------------------------------------------------------------
\begin{frame}{Réflexion et transmission sur une discontinuité}

\small

Exercice complémentaire (correction sur moodle)

On étudie la propagation d'ondes dans un milieu présentant une discontinuité :

$$
(x<0 ) : \quad \rho_0 = \rho_1 ; c_0 = c_1 ; \quad \qquad (x>0 ) : \quad  \rho_0 = \rho_2 ; c_0 = c_2
$$

On suppose que le champ de pression est donné par une loi de la forme suivante :
\begin{equation}
p'(x,t) = \left\{ \begin{array}{ll} 
A e^{i (k_1 x - \omega t)} + B e^{i (-k_1 x - \omega t)} & \quad ( \mbox{ pour }  x<0) \\
C e^{i (k_2 x - \omega t)} & \quad ( \mbox{ pour }  x>0) 
\end{array}
\right.
\label{eq:ABC}
\end{equation}


Montrez que les coefficients de réflexion et de transmission (en intensité acoustique) sont donnés par :

\begin{equation}
R = \frac{|B|^2}{|A|^2} = \left(\frac{\rho_1 c_1 - \rho_2 c_2}{\rho_1 c_1 + \rho_2 c_2}\right)^2,
\quad
T = \frac{|C|^2/\rho_2 c_2}{|A|^2/ \rho_1 c_1}    = \frac{4 \rho_1 c_1\rho_2 c_2}{(\rho_1 c_1 + \rho_2 c_2)^2}.
\label{eq:RT}
\end{equation}

Application numérique : entre l'eau et l'air $T = 0.001  \equiv -30 db$. Très mauvais !

La transmission peut être augmentée avec un dispositif mécanique "adaptateur d'impédance" dont un très bel exemple est constitué par les osselets de l'oreille moyenne.

\begin{figure}
$$
\includegraphics[width=.25\linewidth]{OreilleMoyenne.jpg}
\includegraphics[width=.25\linewidth]{ModeleOreille.png}
$$
\caption{$(a)$ physiologie de l'oreille et $(b)$ Modèle mécanique simplifié des osselets de l'oreille moyenne.}
\end{figure}


\end{frame}



%-----------------------------------------------------------------------------------------
\subsection{Ondes sphériques}
%-----------------------------------------------------------------------------------------
\begin{frame}{Ondes sphériques}
%-----------------------------------------------------------------------------------------

Considérons des ondes décrites en coordonnées sphériques sous la forme $p' = p'(r,t)$ ; 
$\vec{u}'  = u(r,t) \vec{e}_r$.

En utilisant l'expression de l'opérateur Laplacien en coordonnées sphériques, l'équation de Helmholtz s'écrit :

\begin{equation}
	\color{vert}
	\ddpdt{p'} = c_0^2 \nabla^2 p' \equiv \frac{1}{r} \frac{\partial^2}{\partial r^2}\left(r p'\right)
\end{equation}

La solution générale peut s'écrire sous la forme :
$$
p'(r,t) = \frac{f(r-c_0 t)}{r} +  \frac{f(r+c_0t)}{r} 
$$

On reconnait une onde divergente et une onde convergente.


\end{frame}

\begin{frame}{Ondes sphérique : étude de l'onde divergente}
%-----------------------------------------------------------------------------------------
\small

Considérons la solution d'onde sphérique divergente définie par 
$$
p'(r,t) = \frac{f(r-c_0 t)}{r}.
$$

A partir des équations du mouvement on montre que le champ de vitesse associé est donné par $\vec{u}'  = u'(r,t) \vec{e}_r$ avec :

$$
u'(r,t) = \frac{1}{\rho_0 c_0} \left( \frac{f(r-ct)}{r} - \frac{F(r-ct)}{r^2} \right) \quad \mbox{ où } F(r') = \int f(r') d r'.   
$$

Le premier terme est dominant en champ lointain. En ne retenant que ce terme, l'énergie acoustique volumique et l'intensité acoustique sont données par :

$$
\overline{e_{ac}} = \frac{\rho u'^2}{2} + \frac{p'^2}{2 \rho_0 c_0^2} = 
\frac{\overline{f(r-ct)^2}}{\rho_0 c_0^2}
\frac{1}{r^2} 
$$

$$
\overline{\vec{I}} = \overline{p' \vec {u'} } = c_0 \overline{e_{ac}} \vec{e}_r
$$

On constate que l'intensité acoustique (flux surfacique d'énergie acoustique) décroit en $r^{-2}$.

(le flux total $ \int_S \overline{\vec{I}} \cdot \vec{n}  dS$ 
sur une sphère $S$ de rayon $r$ est constant, logique !)




\end{frame}


%-----------------------------------------------------------------------------------------
\begin{frame}{Réflexion sur la pointe d'un tuyau conique}
%-----------------------------------------------------------------------------------------

Considérons une situation correspondant à une onde divergente monochromatique d'amplitude $A$ et une onde convergente monochromatique d'amplitude $B$ :

$$
p'(r,t) =    \frac{B e^{i(kr +\omega t)}}{r} + \frac{A e^{i(kr -\omega t)}}{r} 
$$

La pression $p'(r,t)$ doit rester finie en $r=0$.

Ceci conduit à la condition $A=-B$.

\medskip

Conclusion : dans un tuyau conique, une onde (décrite en coordonnées sphériques) se réfléchit sur la pointe (fermée) en gardant changeant de signe !

\medskip

Conséquence : un tuyau conique (hautbois) a une fréquence fondamentale deux fois plus haute que celle d'un tuyau cylindrique fermé (clarinette). 

Le hautbois sonne donc une octave plus haut que la clarinette !



\end{frame}

\begin{frame}[fragile]{Illustrations multimedia}

\small
Ondes progressive (impulsion)

{\scriptsize
\begin{verbatim}
http://www.animations.physics.unsw.edu.au/jw/waves_superposition_reflection.htm#travelling
\end{verbatim}
}

Onde progressive sinusoidale

{\scriptsize
\begin{verbatim}
http://www.animations.physics.unsw.edu.au/jw/travelling_sine_wave.htm
\end{verbatim}
}


Reflexion sur une discontinuité (ondes sur une corde)

{\scriptsize
\begin{verbatim}
http://www.animations.physics.unsw.edu.au/jw/waves_superposition_reflection.htm#densities
\end{verbatim}
}

Flutes et clarinettes
{\scriptsize
\begin{verbatim}
http://newt.phys.unsw.edu.au/jw/flutes.v.clarinets.html
\end{verbatim}
}

Un bon point de départ pour en savoir plus :

{\scriptsize
\begin{verbatim}
http://www.editions.polytechnique.fr/files/pdf/EXT_0840_2.pdf
\end{verbatim}
}

\end{frame}


